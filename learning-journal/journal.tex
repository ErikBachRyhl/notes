\documentclass[a4paper,12pt]{article}

% Some basic packages
\usepackage[utf8]{inputenc}
\usepackage[T1]{fontenc}
\usepackage{textcomp}
\usepackage[english]{babel}
\usepackage{url}
\usepackage{graphicx}
\usepackage{float}
\usepackage{booktabs}
% \usepackage{enumitem}
\usepackage{enumerate}
\usepackage[colorlinks]{hyperref}

\pdfminorversion=7

% Don't indent paragraphs, leave some space between them
\usepackage{parskip}
\usepackage{changepage}

% Hide page number when page is empty
\usepackage{emptypage}
\usepackage{subcaption}
\usepackage{multicol}
\usepackage[dvipsnames]{xcolor}

% Other font I sometimes use.
% \usepackage{cmbright}

% Math stuff
\usepackage{amsmath, amsfonts, mathtools, amsthm, amssymb}

% Add this line to make equation numbering follow section
\numberwithin{equation}{section}

% Fancy script capitals
\usepackage{mathrsfs}
\usepackage{cancel}
% Bold math
\usepackage{bm}
% Some shortcuts
\newcommand\N{\ensuremath{\mathbb{N}}}
\newcommand\R{\ensuremath{\mathbb{R}}}
\newcommand\Z{\ensuremath{\mathbb{Z}}}
\renewcommand\O{\ensuremath{\emptyset}}
\newcommand\Q{\ensuremath{\mathbb{Q}}}
\newcommand\C{\ensuremath{\mathbb{C}}}

% Easily typeset systems of equations (French package)
\usepackage{systeme}

% Put x \to \infty below \lim
\let\svlim\lim\def\lim{\svlim\limits}

%Make implies and impliedby shorter
\let\implies\Rightarrow
\let\impliedby\Leftarrow
\let\iff\Leftrightarrow
% \let\epsilon\varepsilon

% COURSE SPECIFICS
% GRIFFITHS
\ifdefined\pdfliteral
    \let\griffPdfliteral\pdfliteral
\else \def\griffPdfliteral#1{\special{pdf: literal #1}} \fi

\newcommand\griffr[1][2]{\leavevmode\hbox{\kern1pt\vbox to1ex{}\griffPdfliteral{%
    q 1 J .27 0 0 .27 0 0 cm #1 w
    0 2 m
    0 2 8.1 9.7 9.2 13.2 c
    10.4 16.8 8.4 15.4 8 14.7 c
    7.6 14 6.8 12.6 12 13 c
    17 13.5 14.5 7.8 13.7 6 c
    12.8 4.3 10.3 1.2 11.4 .2 c
    12.6 -.7 18.8 3.6 18.8 3.6 c
    18.8 3.6 l S Q
}\kern6pt}}
\newcommand\hatgriffr{\skew3\hat{\griffr[4]}}

% Add \contra symbol to denote contradiction
\usepackage{stmaryrd} % for \lightning
\newcommand\contra{\scalebox{1.5}{$\lightning$}}

% \let\phi\varphi

% Command for short corrections
% Usage: 1+1=\correct{3}{2}

\definecolor{correct}{HTML}{009900}
\newcommand\correct[2]{\ensuremath{\:}{\color{red}{#1}}\ensuremath{\to }{\color{correct}{#2}}\ensuremath{\:}}
\newcommand\green[1]{{\color{correct}{#1}}}

% horizontal rule
\newcommand\hr{
    \noindent\rule[0.5ex]{\linewidth}{0.5pt}
}

% hide parts
\newcommand\hide[1]{}

% si unitx
\usepackage{siunitx}
\sisetup{locale = FR}

% Environments
\makeatother
% For box around Definition, Theorem, ...
% \usepackage{mdframed}
\usepackage[framemethod=TikZ]{mdframed}

% Custom command to draw a rectangular border around an equation
\setlength{\fboxsep}{5pt}  % Adjust padding inside the box
\usepackage{empheq}
\newcommand*\widefbox[1]{\fbox{\hspace{1em}#1\hspace{1em}}}

\usepackage{environ}  % This package allows for easier custom environment definitions

% Define the custom environment
\NewEnviron{framed}{%
  \begin{empheq}[box=\fbox]{align}
  \BODY
  \end{empheq}
}
% Custom environment to box align equations
% \newenvironment{boxedalign}
%   {\begin{empheq}[box=\fbox]{align}}
%   {\end{align}\end{empheq}}

\newtheorem{thm}{Theorem}[subsection]
\newtheorem{defi}[thm]{Definition}
\newtheorem{lem}[thm]{Lemma}
\newtheorem{ret}{Correction}


\newtheorem*{term}{Terminology}
\newtheorem*{key}{Keywords and Related Concepts}
\newtheorem{lign}[thm]{Equation}
\newtheorem{law}[thm]{Law / Principle}

\usepackage{mathtools}
\DeclarePairedDelimiter\bra{\langle}{\rvert}
\DeclarePairedDelimiter\ket{\lvert}{\rangle}
\DeclarePairedDelimiterX\braket[2]{\langle}{\rangle}{#1\,\delimsize\vert\,\mathopen{}#2}


% \newcounter{theo}[section]
% \renewcommand{\thetheo}{\arabic{section}.\arabic{theo}}

% \mdfsetup{skipabove=1em,skipbelow=0em}
% \theoremstyle{definition}
% \newmdtheoremenv[nobreak=true]{definition}{Definition}
% \newmdtheoremenv[nobreak=true]{theorem}{Theorem}
% \newmdtheoremenv[nobreak=true]{corollary}{Corollary}
% \newmdtheoremenv[nobreak=true]{lemma}{Lemma}

% \newtheorem*{observation}{Observation}
% \newtheorem*{property}{Property}
% \newtheorem*{postulate}{Postulate}
% \newtheorem*{conclusion}{Conlusion}
% \newtheorem*{repitition}{Repitition}
% \newtheorem*{example}{Example}
% \newtheorem*{question}{Question}
% \newtheorem*{intuition}{Intuition}

% End example and intermezzo environments with a small diamond (just like proof
% environments end with a small square)
% \usepackage{etoolbox}
% \AtEndEnvironment{example}{\null\hfill$\diamond$}%
% \AtEndEnvironment{repitition}{\null\hfill$\diamond$}%
% \AtEndEnvironment{opmerking}{\null\hfill$\diamond$}%

% Fix some spacing
% http://tex.stackexchange.com/questions/22119/how-can-i-change-the-spacing-before-theorems-with-amsthm
\makeatletter
\def\thm@space@setup{%
  \thm@preskip=\parskip \thm@postskip=0pt
}


% Exercise 
% Usage:
% \oefening{5}
% \suboefening{1}
% \suboefening{2}
% \suboefening{3}
% gives
% Oefening 5
%   Oefening 5.1
%   Oefening 5.2
%   Oefening 5.3
\newcommand{\exercise}[1]{%
    \def\@exercise{#1}%
    \subsection*{Exercise #1}
}

\newcommand{\subexercise}[1]{%
    \subsubsection*{Exercise \@exercise.#1}
}

\usepackage{xcolor}
\newcommand{\textred}[1]{\textcolor{red}{#1}}

% \lecture starts a new lecture (les in dutch)
%
% Usage:
% \lecture{1}{di 12 feb 2019 16:00}{Inleiding}
%
% This adds a section heading with the number / title of the lecture and a
% margin paragraph with the date.

% I use \dateparts here to hide the year (2019). This way, I can easily parse
% the date of each lecture unambiguously while still having a human-friendly
% short format printed to the pdf.

\usepackage{xifthen}
\def\testdateparts#1{\dateparts#1\relax}
\def\dateparts#1 #2 #3 #4 #5\relax{
    \marginpar{\small\textsf{\mbox{#1 #2 #3 #5}}}
}

\def\@lecture{}%
\newcommand{\lecture}[3]{
    \ifthenelse{\isempty{#3}}{%
        \def\@lecture{Lecture #1}%
    }{%
        \def\@lecture{Lecture #1: #3}%
    }%
    \subsection*{\@lecture}
    \marginpar{\small\textsf{\mbox{#2}}}
}

\def\@chapter{}%
\newcommand{\chapter}[3]{
    \ifthenelse{\isempty{#3}}{%
        \def\@chapter{Chapter #1}%
    }{%
        \def\@chapter{Chapter #1: #3}%
    }%
    \subsection*{\@chapter}
    \marginpar{\small\textsf{\mbox{#2}}}
}

\def\@week{}%
\newcommand{\week}[3]{
    \ifthenelse{\isempty{#3}}{%
        \def\@week{Uge #1}%
    }{%
        \def\@week{Uge #1: #3}%
    }%
    \subsection*{\@week}
    \marginpar{\small\textsf{\mbox{#2}}}
}

% These are the fancy headers
% \usepackage{fancyhdr}
% \pagestyle{fancy}

% LE: left even
% RO: right odd
% CE, CO: center even, center odd
% My name for when I print my lecture notes to use for an open book exam.
% \fancyhead[LE,RO]{Gilles Castel}

% \setlength{\headheight}{5pt}

% % \fancyhead[R]{\@lecture} % Right odd,  Left even
% \fancyfoot[R]{\thepage}  % Right odd,  Left even
% \fancyfoot[C]{\leftmark}     % Center

\makeatother

% Todonotes and inline notes in fancy boxes
\usepackage{todonotes}
\usepackage{tcolorbox}

% Make boxes breakable
\tcbuselibrary{breakable}

% Usage: 
% \begin{correction}
%     Lorem ipsum dolor sit amet, consetetur sadipscing elitr, sed diam nonumy eirmod
%     tempor invidunt ut labore et dolore magna aliquyam erat, sed diam voluptua. At
%     vero eos et accusam et justo duo dolores et ea rebum. Stet clita kasd gubergren,
%     no sea takimata sanctus est Lorem ipsum dolor sit amet.
% \end{correction}
\newenvironment{correction}{\begin{tcolorbox}[
    arc=0mm,
    colback=white,
    colframe=green!60!black,
    title=Correction,
    fonttitle=\sffamily,
    breakable
]}{\end{tcolorbox}}

% Same as 'correction' but color of box is different
\newenvironment{note}{\begin{tcolorbox}[
    arc=0mm,
    colback=white,
    colframe=white!60!black,
    title=Note,
    fonttitle=\sffamily,
    breakable
]}{\end{tcolorbox}}


% Figure support as explained in my blog post.
\usepackage{import}
\usepackage{xifthen}
\usepackage{pdfpages}
\usepackage{transparent}
\newcommand{\incfig}[1]{%
    \def\svgwidth{\columnwidth}
    \import{./figures/}{#1.pdf_tex}
}

% Fix some stuff
% %http://tex.stackexchange.com/questions/76273/multiple-pdfs-with-page-group-included-in-a-single-page-warning
\pdfsuppresswarningpagegroup=1


% My name
\author{Erik Bach Ryhl}


%--------------------------
% PACKAGES
%--------------------------
\usepackage[utf8]{inputenc}
\usepackage[T1]{fontenc}
\usepackage{amsmath, amssymb, amsfonts}
\usepackage{geometry}
\usepackage{enumitem}
\usepackage{xcolor}
\usepackage{hyperref}
\usepackage{graphicx}
\usepackage{titlesec}
\usepackage{fancyhdr}
\usepackage[normalem]{ulem}

%--------------------------
% PAGE SETUP
%--------------------------
\geometry{margin=1in}
\pagestyle{fancy}
\fancyhf{}
\rhead{\thepage}
\lhead{Learning Journal}

%--------------------------
% SECTION FORMATTING
%--------------------------
\titleformat{\section}[block]{\large\bfseries}{Session \thesection:}{1em}{}
\titleformat{\subsection}[block]{\bfseries}{\thesubsection}{1em}{}
\setcounter{secnumdepth}{2}

%--------------------------
% CUSTOM ENVIRONMENTS
%--------------------------
\newenvironment{deff}[1][Definition]{\par\vspace{1em}\noindent\textbf{#1.}\hspace{0.5em}}{\par\vspace{0.1em}}

\newenvironment{rem}[1][Remark]{\par\vspace{1em}\noindent\textit{#1.}\hspace{0.5em}\begingroup\itshape}{\endgroup\par\vspace{0.1em}}

% Explanation of the custom environment:
% 1. \newenvironment{definition}[1][Definition]: This creates a new environment named "definition" with an optional argument. The default title is "Definition."
% 2. \par\vspace{1em}: Adds vertical space before the environment begins.
% 3. \noindent\textbf{#1.}: Formats the title (e.g., "Definition.") as bold text.
% 4. \hspace{0.5em}: Adds a small space between the title and the definition text.
% 5. The closing curly braces define the formatting after the environment ends, including vertical spacing.


%--------------------------
% DOCUMENT
%--------------------------
\begin{document}

\begin{center}
    {\Huge \textbf{Learning Journal}} \\
    \vspace{0.5em}
    {\large Organized by Sessions with Topic Summaries}
\end{center}
\vspace{1em}
\hrule
\vspace{2em}

%--------------------------
% TEMPLATE STRUCTURE
%--------------------------
\section*{Instructions}
\begin{itemize}
    \item Use this document to track your learning sessions.
    \item Each session entry should include: \begin{enumerate}
        \item Topics covered (briefly).
        \item Key insights or definitions.
        \item Problems attempted (with references or solutions if necessary).
        \item Questions or areas for follow-up.
    \end{enumerate}
    \textit{Make Anki flashcards of whatever makes sense. Even topics if you'd like, just to leverage their spaced repitition algorithm.}
    \item After completing a topic, write a summary essay, incorporating insights from session notes and solved examples.
\end{itemize}
\newpage

%--------------------------
% RESOURCES AND LEARNING PLANS
%--------------------------
\section*{Resources}
\subsection*{Physics}
\begin{itemize}
    \item Eigenchris
    \item MIT OpenCourseware
    \item Richard Behiel
    \item Steve Brunton
\end{itemize}
\subsection*{Math}
\begin{itemize}
    \item VisualMath on YT \begin{itemize}
        \item Lectures on quantum topology without topology seem really cool! His course notes are also free.
        \item Many overview videos
        \item Algebraic Topology
        \item Algebraic Geometry
        \item etc.
    \end{itemize}
    \item The Bright Side of Mathematics
\end{itemize}
\newpage

\section*{Learning Plan}
\textit{This is a first pass on some topics I find interesting. I don't need to understand everything the first time. I don't need to go through all the subtopics the first time. I need to understand the main idea, understand why and when it is useful, and try my hand at some basic problems to use it myself. I can always pick up these topics again since I keep a record here. Just remember to jot down your questions, the things you don't understand (and why) and what you'd like to investigate further in the future. You can \textbf{always} come back!}
\textit{The below learning plan is not set in stone. Be open for modifications and new ideas. Follow your curiosity.}
\subsection*{Calculus of Variations in Physics - \textred{21th December}}   
\begin{itemize}
    \item \sout{Understand the derivation of the Euler-Lagrange Equations}
    \item Solve the QM problem in Hand and Finch (17th December)
    \item Answer questions below and synthesize and finalize notes on the topic (20th December)
\end{itemize}
\textbf{Questions}
\begin{itemize}
    \item How is it used in modern physics today?
    \item Does one use different "actions" when working on separate problems. If so, could one find a problem "midway" between those problems and see what the action looks like there? Maybe smoothly interpolate an action between these two problems to gain a deeper understanding of how and why they need different descriptions. Maybe find a general description which they are both special cases of. 
\end{itemize} 
\subsection*{Special Relativity, Classical Field Theory and Tensors - 28th December}
\begin{itemize}
    \item Susskind Book (26rd December)
    \item Learn some tensor notation from Tong Notes
    \item Solving some basic problems with 4-vectors (27th December)
    \item Synthesize and finalise (28th December)
\end{itemize}
\subsection*{Floquet Theory - 31st December}
\subsection*{Schuller's Course and self-defined problems - 31st Janurary}
\subsection*{Linear Algebra Refresher - 7th February}
\subsection*{Complex Analysis Basics - 14th February}
\subsection*{MIT Quantum Mechanics Course - 7th February}
\subsection*{Misc. things}
\begin{itemize}
    \item Green's Function solution of Poisson's equation to get the electric potential from Griffiths
\end{itemize}
\newpage

%--------------------------
% SESSIONS
%--------------------------
\section{Session Date: 14th December, 2024}
\subsection*{Main Topic: Set Theory}
\subsection*{Resource: Geometrical Anatomy of Physics}
\subsection*{Topics Covered}
\begin{itemize}
    \item Space
    \item Maps
    \item Domain
    \item Codomain
    \item Image
    \item Preimage
    \item Bijection
    \item Inverse
    \item Equivalence Relation
    \item Equivalence Class
    \item Quotient Space
    \item What amplitudes are
    \item What a Positive Grassmannian is
\end{itemize}

\subsection*{Key Insights}
\subsubsection*{Definitions from Schuller's lecture series} 
\begin{deff}
    A \textbf{space} is a set with some underlying structure. We often study structure-preserving maps between such spaces.
\end{deff}
\begin{deff}
    A \textbf{map} is a relation between to sets. More formally, we can write that given two sets \(A\) and \(B\), a map \(\phi : A \to B\) is a relation such that for each element \(a \in A\) there exists exactly one element \(b \in B\) such that \(\phi (a , b)\). We write that \(a \mapsto \phi(a)\).   
\end{deff}
\begin{deff}
    Here, \(A\) is the \textbf{domain} and \(B\) is the \textbf{codomain}. The codomain is also called the target.
\end{deff}
\begin{deff}
    The \textbf{image} of a set \(C \subseteq A\) under a map \(\phi\)  is the set one gets, which will be a subset of (or equal to) the codomain, by collecting everything that \(\phi \)  maps to when applied to \(C\). We write \(\phi (C) \equiv \rm{im}_\phi(C) \coloneqq \left\{ \phi (c)\ |\ c \in C \right\}\). 
\end{deff}
\begin{deff}
    The \textbf{preimage} is the set, which will be a subset of (or equal to) the domain, one gets by considering which elements in the domain one has to apply \(\phi \) to to get certain elements in the codomain. Let for example \(V \subseteq B\), then \(\rm{preim}_{\phi}(V) \coloneqq \left\{ a \in A\ | \  \phi (a) \in V\right\}  \)
\end{deff}
\begin{rem}
    The inverse is only defined for bijections, but the preimage is defined for all maps, and we will often meet it in topology! I was confused at first as to why we know that \(\text{preim}_{\phi }(B) = A\) without requiring surjectiveness, but this is because when we write \(\phi : A \to B\) we are already stating that \(\phi \) is applied to, or at least makes sense to apply to, all of \(A\). Now the image might not be all of \(B\) (it just "lives" in \(B\)), but the preimage of \(B\) is the set of all of the values in the domain which under the map \(\phi\) ends up in \(B\) - but that is of course all of \(A\), since from the definition, applying \(\phi\) to any element in \(A\) it will end up \(B\).     
\end{rem}
    \begin{deff}
        A map is \textbf{surjective} if \(\phi (A) \equiv \text{im}_{\phi }(A) = B\) - that is, if all of \(B\) is "hit" by applying \(\phi \) to all of \(A\). A map is \textbf{injective} if for \(a_1, a_2 \in A\) we have that \(\phi (a_1) = \phi (a_2) \implies a_1 = a_2\). 
        
        The most important notion: A map is called \textbf{bijective} if it is both surjective and injective. 
\end{deff}
\begin{deff}
    When a map is bijective, then a unique \textbf{inverse} exists. This is the map such that \(\phi^{-1}  \circ \phi^ = \rm{id}_A\) while \(\phi \circ \phi ^{-1} = \rm{id}_B\). In other words, it "undoes" a mapping. Reading \(\circ\) as "after" helps to learn the order of application.    
\end{deff}
\begin{rem}
    Generically, if there exists one bijection between sets, then there exists many. A bijection is just a "pairing up" of elements - if you can come up with one way, then you can certainly come up with many (unless you try to design a counterexample I guess).
\end{rem}
\begin{deff}
    If there exists any bijection between two sets \(A\) and \(B\) then we say that they are (set-theoretically) isomorphic ("of the same shape"). We write \(A \cong_{set}  B\).   
\end{deff}
\begin{deff}
    An \textbf{equivalence relation} is any relation between elements in a set which is both \textit{reflexive}, \textit{symmetric} and \textit{transative}. Letting \(\sim\) denote the relation, we write these as \begin{align*}
        &a \sim a\quad \text{(reflexive)}\\
        &a \sim b \iff b \sim a\quad \text{(symmetric)}\\
        &a \sim b \wedge b \sim c \implies a \sim c\quad \text{(transative)}
    \end{align*} 
    We denote all of the elements of \(A\) which are equivalent to some \(m \in A\) under the given equivalence relation as \([m] \coloneqq \left\{ n \in A\ | \ n \sim m \right\} \).
\end{deff}
\begin{rem}
    I am very proud since I was able to prove the following: \begin{enumerate}[label=\roman*)]
        \item \(a \in [m] \implies [a] = [m]\)
        \item Either \([a] = [m]\) or \([a] \cap [m] = \emptyset \) 
    \end{enumerate}
    The first of these results imply that any member of an equivalence class equally well represents the whole class. The second one implies that an equivalence relation completely splits the set \(A\) into disjoint equivalence classes - we say that it "partitions" the set. 
\end{rem}
\begin{deff}
    The set of all equivalence classes formed by applying the equivalence relation \(\sim\) to the set \(A\) is called the \textbf{quotient space} and is written as \(A\setminus\sim\). Intuitively, the quotient space is what you get when you sort your large set \(A\) into smaller sets by using some rules defined by the given equivalence relation. Examples of an equivalence relation is modulo divison by some prime number. One can then take say the quotient space \(\mathbb{Z} \setminus \rm{mod}\ p\).
\end{deff}
\subsubsection*{Takeaways}
A map \(\phi : A \to B\) applies, by definition, to \textbf{all} elements in the domain \(A\). This crucially does not necessarily mean that all elements in \(B\) has a corresponding element in \(A\) under this map; this property is exactly \textit{surjectivity}. 

The set obtained from applying the map to the entire domain is the \textit{image} of the map. If the codomain and the image are equal, then the map is surjective. Thus we can always redefine the codomain to be the image and then the map becomes surjective. But this is often not very interesting.

But the fact that the map is understood to apply to all the elements of the domain is needed to understand why for the map \(\phi : A \to B\) we find that \begin{align*}
    \text{preim}_{\phi}(B) = A 
\end{align*} 

where for some \(V \subseteq B\) we define the preimage as \begin{align*}
    \text{preim}_{\phi}(V) = \left\{ a \in A\ |\ \phi (a) \in V \right\}  
\end{align*} 

\subsubsection*{Amplitudes and the positive Grassmannian}
Studying amplitudes is about studying what we expect to happen when fundamental particles interact at very high energies, like when being smashed into each other at the LHC. Physicists then calculate the probabilities related with the particle scattering in different directions with different energies and momenta. These probabilites are precisely the "amplitudes" in "scattering" and "scattering amplitudes". The reason why this is interesting is because if we want to know if our theory is right - or if we want to know precisely when and where it is not - then we need to have very precise expectations from experiments such that we know when they deviate. 

Studying amplitudes is therefore very much at the heart of our most fundamental understanding of nature, and it actually sounds really exciting.

The Grassmannian is a way to group and classify subspaces embedded in larger spaces. For example, \(\rm{Gr}(k, d)\) is the collection of all \(k\)-dimensional subspaces going through the origin in the larger \(d\)-dimensional space. I think. The positive Grassmannian is the subspace of the Grassmannian which only has positive minors along (\textred{all or certain}) axes. I think minors are subdeterminants or something? I'm note sure. But the intuition is that if we are considering the space of all lines in 3D going through the origin \(\rm{Gr}(1, 3)\), then the positive Grassmannian would be only the lines with positive slope. The Grassmannian kind of "keeps track" of all these distinct geometric objects (lines with different slopes and directions) by only having them as points. The full Grassmannian just discussed would uniquely identify each point on the upper hemisphere with a line (expect for lines going through the "equator" in the \((x, y)\)-plane, if the hemisphere is formed by cutting a sphere in two in the \((x,y)\)-plane).

Apparently, great advances were made in calculating scattering amplitudes by using the positive Grassmannian. And this was just around 10 years ago - so it is still relatively new!. Calculating these amplitudes was (and probably still is) usually is done by adding hundreds of Feynman diagrams for even the simplest calculations - and many thousands for a bit more interesting interactions. And most of these terms sum to zero or something very concise, which is very difficult to understand from the size of the sum. In other words, a lot of redundancies are inherent in the Feynman diagram way of calculating scattering amplitudes, and people are working on more direct ways of doing it since there must be a reason for why many answers come out so beautiful and concise even though the actual calculation is the most messy thing ever.

\subsection*{Problems Attempted}
\begin{enumerate}
    \item Proving the statements regarding equivalence classes (huge victory!)
\end{enumerate}

\subsection*{Follow-Up Questions}
\begin{itemize}
    \item Do a short write up of the proofs above. Remind yourself of the general proof-technique to use if one wants to show a "either - or" statement. Reminder: show \(p \implies \neg q\) because then \(\neg(\neg q) \implies \neg p\) (through contrapositive) which is the same as \(q \implies \neg p\). 
\end{itemize}

\newpage
\section{Session Date: 15th December, 2024}
\subsection*{Main Topic: Electromagnetism}
\subsection*{Topics Covered}
\begin{itemize}
    \item A small window into amplitudes (Cheung Lecture)
    \item Fields from electric monopoles, magnetic dipoles and the classical electron radius
\end{itemize}

\subsection*{Key Insights}
\paragraph{The Classical Electron Radius} is the radius one gets if setting the rest mass of the electron equal to the energy stored in the electric field outside that same radius. 

\textbf{What assumption in the calculation makes it so wrong?}

The assumption being made here is that an electron is a finitely sized, spherically, uniformly charged object. That is simply not the case, and high-energy scattering finds that it behaves like a point-particle down to \(10^{-18}\). One needs QED and renormalization as well as effective field theories to properly explain the behaviour of the electron. A problem that arises with the finite size is that there is no explanation for why the energy doesn't radiate. At the scale of the supposed radius, quantum effects are significant and the electron's energy should fluctuate and self-interact with the field, I think. One can use the \textbf{Compton Wavelength} to get a feel for when quantum effects become significant: \begin{align*}
    \lambda _c = \frac{h}{m_e c^{2} } \approx 2.43 \cdot 10^{-12}\ \rm{m}
\end{align*}  
\paragraph{Infinity of a point particle} 
The problem with a point particle in classical electrodynamics is infinities. When doing the assignment we also found that the closer you integrate both the electric field and the magnetic field to the electric charge and the magnetig dipole respectively, the more energy you find - and this goes towards infinity as the integration approaches all space (by making the excempt sphere around the mono/dipole smaller). Take the electric point charge. This infinite energy makes sense since we can imagine that we take the total charge of the point charge and split it up into smaller, less charged point particles. Now, gathering those point particles in the same point to get the total charge would require us to work an infinite amount against the fields, since the strength of the field becomes infinite as the point charges go to sit on top of each other. The same principle applies with the magnetic dipole. Given a perfect magnetic dipole \(\mathbf{m} = I \int d \mathbf{a}\), we see that we need an infinite amound of current to get a finite \(m\) since \(\int d \mathbf{a}\) is zero for a perfect dipole. But running an infinite current also requires an infinite amount of energy. So this divergence of energy makes sense from Maxwell's theories as well. 

\subsection*{Problems Attempted}
\begin{enumerate}
    \item Calculating the "classical electron radius"
\end{enumerate}

\subsection*{Follow-Up Questions}
\begin{itemize}
    \item Why do we expect Hamilton's principle to work for scalar fields in arbritrary dimensions? We know that it works well in 3 dimensions because we can do experiments - but is it a leap of faith to do it in higher dimensions, or do we have some clue to its validity even in higher dimensions? What if the look of the least action principle is a special case in 3D, and there is a more general mathematical principle - a geometry maybe - which underlies the whole thing in arbritrary \(D\)-dimensional space?
    
    \textbf{Answer:} All of the general results shown from the least action principle (Noether's theorem, path integral formulation) are not sensitive to the dimensionality of the problem. There seems to be nothing special about the principle's application in 3D. Once setting up the fields and a Lagrangian density, a stationary action results in a \textit{local} differential equation which motion must conform to. There is so much more to be said about all of these things, but it will have to wait.
\end{itemize}

\newpage
\section{Session Date: 14th December, 2024}
\subsection*{Main Topic: Electromagnetism}
\subsection*{Topics Covered}
\begin{itemize}
    \item Total momentum in fields from momentum density
    \item Angular momentum in the fields from momentum density
    \item Complex current from complex impedances
\end{itemize}

\subsection*{Key Insights}
\subsubsection*{Takeaways}
If one has a field whose magnitude only depends on \(r\) but whose direction is given by \(\hat{\phi}\), then any integral one full revolution in the \(\phi \)-direction will of course give \(\mathbf{0}\). This makes great sense both when you think about it geometrically (you will add up as many components in one direction as in any other, hence summing up to zero) or if you write it out in cartesian components you will integrate \(\sin \phi \) and \(\cos \phi \) around a full period which gives zero.

\subsection*{Follow-Up Questions}
\begin{itemize}
    \item How does one derive rest energy? I don't remember where it comes from.
\end{itemize}

\newpage
\section{Session Date: 17th December, 2024}
\subsection*{Main Topic: Electromagnetism}
\subsection*{Topics Covered}
\begin{itemize}
    \item Maxwell's equations in terms of potentials only
    \item Gauge Transformations
    \item 4-vector notation
    \item d'Alembert operator
    \item Retarded potentials
\end{itemize}

\subsection*{Key Insights}
\subsubsection*{Derivation Recap}
\begin{enumerate}[label=\roman*]
    \item) $\nabla \cdot \mathbf{E} = \frac{\rho}{\epsilon_0}$
    \item) $\nabla \cdot \mathbf{B} = 0$
    \item) $\nabla \times \mathbf{E} = - \frac{\partial \mathbf{B}}{\partial t}$
    \item) $\nabla \times \mathbf{B} = \mu_0 \mathbf{J} + \mu_0 \epsilon _0 \frac{\partial \mathbf{E}}{\partial t}$
\end{enumerate}

ii) allows us to write \begin{align*}
    \boxed{\mathbf{B} = \nabla \times \mathbf{A}}
\end{align*}since the divergence of any curl is zero. This allows us to write \begin{align*}
    \nabla \times \mathbf{E} &= - \nabla \times \frac{\partial \mathbf{A}}{\partial t} \\
    &\implies \nabla \times \left( \mathbf{E} + \frac{\partial \mathbf{A}}{\partial t}  \right) = 0
\end{align*}

And since the curl of any gradient is zero too, we know that we can write the above as the (negative) gradient of a potential too:
\begin{align*}
    - \nabla V = \mathbf{E} +\frac{\partial \mathbf{A}}{\partial t}
\end{align*}
or \begin{align*}
    \boxed{\mathbf{E} = -\nabla V - \frac{\partial \mathbf{A}}{\partial t}}
\end{align*}
such that Gauss' law becomes \begin{align*}
    \nabla \cdot \left( -\nabla V - \frac{\partial \mathbf{A}}{\partial t}  \right) = - \nabla^{2} V - \nabla \cdot \frac{\partial \mathbf{A}}{\partial t} = \frac{\rho}{\epsilon _0}\\
    \implies \nabla ^{2} V + \nabla \cdot \frac{\partial \mathbf{A}}{\partial t} = -\rho  /\epsilon_0
\end{align*}

In the static case (\(\partial _t \mathbf{A} = 0\) ) this reduces to Laplace's equation \begin{align*}
    \nabla ^{2} V = - \frac{\rho}{\epsilon _0}
\end{align*}

We can also rewrite the Ampére-Maxwell law, keeping the source on the right and moving anything else (the fields or potentials) to the left: \begin{align*}
    \nabla \times \mathbf{B} - \frac{1}{c^{2}} \frac{\partial \mathbf{E}}{\partial t} = \mu _0 \mathbf{J}
\end{align*}
which in terms of the potentials gives \begin{align*}
    &\nabla \times \left( \nabla \times \mathbf{A}\right) - \frac{1}{c^{2} } \frac{\partial}{\partial t} \left( -\nabla V - \frac{\partial \mathbf{A}}{\partial t}  \right)  = \mu _0 \mathbf{J}\\
    =\ &\nabla (\nabla \cdot \mathbf{A}) - \nabla ^{2} \mathbf{A} + \frac{1}{c^{2}}\nabla \left( \frac{\partial V}{\partial t}  \right) + \frac{1}{c^{2}}\frac{\partial^{2} \mathbf{A}}{\partial t^{2} } \\
    =\ &\nabla \left( \nabla \cdot \mathbf{A} + \frac{1}{c^{2} } \frac{\partial V}{\partial t} \right) - \left( \nabla ^{2} - \frac{1}{c^{2}} \frac{\partial^{2} }{\partial t^{2} }  \right)\mathbf{A}
\end{align*}
Defining the d'Alembertian as \begin{align*}
    \square^{2} \equiv \nabla ^{2} - \frac{1}{c^{2} }\frac{\partial ^{2}}{\partial t^{2} }  
\end{align*}
and letting \begin{align*}
    L \equiv \nabla \cdot \mathbf{A} + \frac{1}{c^{2} } \frac{\partial V}{\partial t} 
\end{align*}
we can succintly write the Ampére-Maxwell law as: \begin{align*}
    \boxed{\square^{2} \mathbf{A} - \nabla L = - \mu _0 \mathbf{J}}
\end{align*}

Notice how we can use this in Gauss' law: \begin{align*}
    &\nabla ^{2} V + \nabla \cdot \frac{\partial \mathbf{A}}{\partial t} = -\rho  /\epsilon_0\\
    \implies &\nabla ^{2} V - \frac{1}{c^{2} }\frac{\partial^{2}  V}{\partial t^{2} }  + \frac{\partial }{\partial t}\left( \nabla \cdot \mathbf{A} + \frac{1}{c^{2} }\frac{\partial V}{\partial t}  \right)   = -\rho  /\epsilon_0\\
    =\ & \boxed{\square^{2}V + \frac{\partial L}{\partial t} = - \rho /\epsilon _0}
\end{align*}

Such that Maxwell's equation in terms of potentials become 



\subsection*{Follow-Up Questions}
\begin{itemize}
    \item How come the retarded potential formulation is not equivalent to putting a heaviside inside the integration.
    \item Why will you get all modes below the cutoff frequency inside wave guides?
    \item Show what Gauge invariance means
    \item Do an example from the blackboar
    \item Rederive boundary conditions for the fields (both free and in matter)
    \item Learn more about the 4-vector formulation and how \(g_{\mu \nu}\) can "raise or lower indecies"
    \item Lorentz tansformationts as rotations in 4-space (Mogen's notes)
    \item Coloumb Gauge (\(\nabla \cdot \mathbf{A}\) )
    \item Lorenz Gauge (\(L = 0\))
    \item Walk through the full retarded potential derivation.  
\end{itemize}

\newpage
\section{Session Date: 18th December, 2024}
\subsection*{Main Topic: Electromagnetism}
\subsection*{Topics Covered}
\paragraph{Cool way to do propagation of errors}
Variance is equal to \(\sigma^{2}\). If our function is \begin{align*}
    f = xy
\end{align*}
the law of propagation of errors gives \begin{align*}
    V(f) &= \frac{\partial f}{\partial x}V(x) + \frac{\partial f}{\partial y} V(y)\\
    &= y V(x) + xV(y)\\
\end{align*}
such that \begin{align*}
    \left( \frac{\sigma _f}{f} \right)^{2} = \left( \frac{\sigma_x}{x} \right) ^{2} + \left( \frac{\sigma _y}{y} \right) ^{2} 
\end{align*}
which also works for fractions.

\paragraph{Understood how we can conclude that the \(\mathbf{E}\) and \(\mathbf{B}\) fields are in phase for monochromatic plane waves.}
We derive that \begin{align*}
    k(\tilde{E}_0)_x &= \omega (\tilde{B}_0)_y\\
    -k(\tilde{E}_0)_y &= \omega (\tilde{B}_0)_x
\end{align*}
But since \(k\) and \(\omega \) are real, the only way that these scalings can hold all along the wave, then the waves have to hit zero and peak at the same time, which means that they are in phase! Since we from above have the neat writing that \begin{align*}
    \tilde{\mathbf{B}}_0 = \frac{k}{\omega} (\hat{z} \times \tilde{\mathbf{E}}_0) = \frac{1}{c}(\hat{z} \times \tilde{\mathbf{E}}_0)
\end{align*}
taking the modulus we get that \begin{align*}
    B_0 = \frac{k}{\omega }E_0 = \frac{1}{c} E_0
\end{align*}
even for the real wave.


\end{document}
