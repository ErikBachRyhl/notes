\section{Session Date: 17th December, 2024}
\subsection*{Main Topic: Electromagnetism}
\subsection*{Topics Covered}
\begin{itemize}
    \item Maxwell's equations in terms of potentials only
    \item Gauge Transformations
    \item 4-vector notation
    \item d'Alembert operator
    \item Retarded potentials
\end{itemize}

\subsection*{Key Insights}
\subsubsection*{Derivation Recap}
\begin{enumerate}[label=\roman*]
    \item) $\nabla \cdot \mathbf{E} = \frac{\rho}{\epsilon_0}$
    \item) $\nabla \cdot \mathbf{B} = 0$
    \item) $\nabla \times \mathbf{E} = - \frac{\partial \mathbf{B}}{\partial t}$
    \item) $\nabla \times \mathbf{B} = \mu_0 \mathbf{J} + \mu_0 \epsilon _0 \frac{\partial \mathbf{E}}{\partial t}$
\end{enumerate}
In electrostatics, we had \(\nabla \times \mathbf{E} = \mathbf{0}\) which allowed us to write \(\mathbf{E} = - \nabla V\). But in electrodynamics, the curl of \(\mathbf{E}\) isn't zero. But we can still get to a potential formulation:  

ii) still allows us to write \begin{align*}
    \boxed{\mathbf{B} = \nabla \times \mathbf{A}}
\end{align*}since the divergence of any curl is zero. This allows us to write \begin{align*}
    \nabla \times \mathbf{E} &= - \nabla \times \frac{\partial \mathbf{A}}{\partial t} \\
    &\implies \nabla \times \left( \mathbf{E} + \frac{\partial \mathbf{A}}{\partial t}  \right) = 0
\end{align*}

And \textit{now}, since the curl of any gradient is zero too, we know that we can write the above as the (negative) gradient of a potential too:
\begin{align*}
    - \nabla V = \mathbf{E} +\frac{\partial \mathbf{A}}{\partial t}
\end{align*}
or \begin{align*}
    \boxed{\mathbf{E} = -\nabla V - \frac{\partial \mathbf{A}}{\partial t}}
\end{align*}
such that Gauss' law becomes \begin{align*}
    \nabla \cdot \left( -\nabla V - \frac{\partial \mathbf{A}}{\partial t}  \right) = - \nabla^{2} V - \nabla \cdot \frac{\partial \mathbf{A}}{\partial t} = \frac{\rho}{\epsilon _0}\\
    \implies \nabla ^{2} V + \nabla \cdot \frac{\partial \mathbf{A}}{\partial t} = -\rho  /\epsilon_0
\end{align*}

In the static case (\(\partial _t \mathbf{A} = 0\) ) this reduces to Laplace's equation \begin{align*}
    \nabla ^{2} V = - \frac{\rho}{\epsilon _0}
\end{align*}

We can also rewrite the Ampére-Maxwell law, keeping the source on the right and moving anything else (the fields or potentials) to the left: \begin{align*}
    \nabla \times \mathbf{B} - \frac{1}{c^{2}} \frac{\partial \mathbf{E}}{\partial t} = \mu _0 \mathbf{J}
\end{align*}
which in terms of the potentials gives \begin{align*}
    &\nabla \times \left( \nabla \times \mathbf{A}\right) - \frac{1}{c^{2} } \frac{\partial}{\partial t} \left( -\nabla V - \frac{\partial \mathbf{A}}{\partial t}  \right)  = \mu _0 \mathbf{J}\\
    =\ &\nabla (\nabla \cdot \mathbf{A}) - \nabla ^{2} \mathbf{A} + \frac{1}{c^{2}}\nabla \left( \frac{\partial V}{\partial t}  \right) + \frac{1}{c^{2}}\frac{\partial^{2} \mathbf{A}}{\partial t^{2} } \\
    =\ &\nabla \left( \nabla \cdot \mathbf{A} + \frac{1}{c^{2} } \frac{\partial V}{\partial t} \right) - \left( \nabla ^{2} - \frac{1}{c^{2}} \frac{\partial^{2} }{\partial t^{2} }  \right)\mathbf{A}
\end{align*}
Defining the d'Alembertian as \begin{align*}
    \square^{2} \equiv \nabla ^{2} - \frac{1}{c^{2} }\frac{\partial ^{2}}{\partial t^{2} }  
\end{align*}
and letting \begin{align*}
    L \equiv \nabla \cdot \mathbf{A} + \frac{1}{c^{2} } \frac{\partial V}{\partial t} 
\end{align*}
we can succintly write the Ampére-Maxwell law as: \begin{align*}
    \boxed{\square^{2} \mathbf{A} - \nabla L = - \mu _0 \mathbf{J}}
\end{align*}

Notice how we can use this in Gauss' law: \begin{align*}
    &\nabla ^{2} V + \nabla \cdot \frac{\partial \mathbf{A}}{\partial t} = -\rho  /\epsilon_0\\
    \implies &\nabla ^{2} V - \frac{1}{c^{2} }\frac{\partial^{2}  V}{\partial t^{2} }  + \frac{\partial }{\partial t}\left( \nabla \cdot \mathbf{A} + \frac{1}{c^{2} }\frac{\partial V}{\partial t}  \right)   = -\rho  /\epsilon_0\\
    =\ & \boxed{\square^{2}V + \frac{\partial L}{\partial t} = - \rho /\epsilon _0}
\end{align*}

Such that Maxwell's equation in terms of potentials become 



\subsection*{Follow-Up Questions}
\begin{itemize}
    \item Why will you get all modes above the cutoff frequency inside wave guides? \textbf{Answer}: We show mathematically that the most general solution includes all frequencies of \(\omega = m \pi / a\) with \(m \in \mathbb{Z} \) or something. So I guess there's no reason to assume that they shouldn't be there, since they can? Still, a vague answer.
\end{itemize}
