\textit{The below learning plan is not set in stone. Be open for modifications and new ideas. Follow your curiosity.}
\subsection*{Calculus of Variations in Physics - \textred{21th December}}   
\begin{itemize}
    \item \sout{Understand the derivation of the Euler-Lagrange Equations}
    \item Solve the QM problem in Hand and Finch (17th December)
    \item Answer questions below and synthesize and finalize notes on the topic (20th December)
\end{itemize}
\textbf{Questions}
\begin{itemize}
    \item How is it used in modern physics today?
    \item Does one use different "actions" when working on separate problems. If so, could one find a problem "midway" between those problems and see what the action looks like there? Maybe smoothly interpolate an action between these two problems to gain a deeper understanding of how and why they need different descriptions. Maybe find a general description which they are both special cases of. 
\end{itemize} 
\subsection*{Special Relativity, Classical Field Theory and Tensors - 28th December}
\begin{itemize}
    \item Susskind Book (26rd December)
    \item Learn some tensor notation from Tong Notes
    \item Solving some basic problems with 4-vectors (27th December)
    \item Synthesize and finalise (28th December)
\end{itemize}
\subsection*{Floquet Theory - 31st December}
\subsection*{Schuller's Course and self-defined problems - 31st Janurary}
\subsection*{Linear Algebra Refresher - 7th February}
\subsection*{Complex Analysis Basics - 14th February}
\subsection*{MIT Quantum Mechanics Course - 7th February}
\subsection*{Misc. things}
\begin{itemize}
    \item Green's Function solution of Poisson's equation to get the electric potential from Griffiths
\end{itemize}