\section{Session Date: 25th December, 2024}
\subsection*{Main Topic: Electromagnetism}
\subsection*{Topics Covered}
\begin{itemize}
    \item Gauge Transformations
\end{itemize}

\subsection*{Key Insights}
\subsubsection*{Takeaways}
\paragraph{Divergence and Curl from symmetry.} If something has a radial symmetry (for example a radial current density), it \textit{cannot} have a preferred direction in space (at least in \(\mathbb{R}^3\)). And then it cannot have a curl, since a non-zero curl requires something to curl \textit{around}. You can also say that anything that has a radial symmetry necessarily has a non-zero divergence, and hence a radially symmetric field cannot represent a magnetic field, since it has \(\nabla \cdot \mathbf{B} = 0\). 

\subsection*{Problems Attempted}
\paragraph{10.7 in Griffiths.} Remember that \begin{align*}
    \nabla \left( \frac{1}{\griffr[2] } \right) = - \frac{\hatgriffr }{\griffr[2] ^{2} }
\end{align*}
and that \begin{align*}
    \nabla \cdot \left( \frac{\hatgriffr }{\griffr[2] ^{2} } \right) = 4 \pi \delta ^{3}(\mathbf{r})
\end{align*}

\paragraph{10.9 in Griffiths.} Notice that \begin{align*}
    \mathbf{v} \cdot \left( \mathbf{v} \times (\nabla \times \mathbf{A}) \right) = \mathbf{0}
\end{align*}
and that \begin{align*}
    \mathbf{v} \cdot \nabla V = v_{i} \partial _i V = \frac{dx_i}{dt} \partial _i V = \frac{dV}{dt}
\end{align*}

