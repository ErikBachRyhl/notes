\section{Session Date: 14th December, 2024}
\subsection*{Main Topic: Set Theory}
\subsection*{Resource: Geometrical Anatomy of Physics}
\subsection*{Topics Covered}
\begin{itemize}
    \item Maps and bijections
    \item Maps
    \item Domain
    \item Codomain
    \item Image
    \item Preimage
    \item Equivalence Relation
    \item Equivalence Class
\end{itemize}

\subsection*{Key Insights}
\subsubsection*{Definitions} 
\subsubsection*{Theorems}
\subsubsection*{Takeaways}
A map \(\phi : A \to B\) applies, by definition, to \textbf{all} elements in the domain \(A\). This crucially does not necessarily mean that all elements in \(B\) has a corresponding element in \(A\) under this map; this property is exactly \textit{surjectivity}. 

The set obtained from applying the map to the entire domain is the \textit{image} of the map. If the codomain and the image are equal, then the map is surjective. Thus we can always redefine the codomain to be the image and then the map becomes surjective. But this is often not very interesting.

But the fact that the map is understood to apply to all the elements of the domain is needed to understand why for the map \(\phi : A \to B\) we find that \begin{align*}
    \text{preim}_{\phi}(B) = A 
\end{align*} 

where for some \(V \subseteq B\) we define the preimage as \begin{align*}
    \text{preim}_{\phi}(V) = \left\{ a \in A\ |\ \phi (a) \in V \right\}  
\end{align*} 

\subsection*{Problems Attempted}
\begin{enumerate}
    \item \textit{Problem statement or reference.}
    \item \textit{Solution (include partial work if needed).}
\end{enumerate}

\subsection*{Follow-Up Questions}
\begin{itemize}
    \item \textit{Write down any gaps in understanding or questions to revisit.}
\end{itemize}
