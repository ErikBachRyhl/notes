\section{Session Date: 23rd December, 2024}
\subsection*{Main Topic: Electromagnetism, Calculus of Variations, Fourier Transform}
\subsection*{Resource: Griffiths, Hand \& Finch, YT}
\subsection*{Topics Covered}
\begin{itemize}
    \item \textit{List key topics or subtopics here.} 
\end{itemize}

\subsection*{Key Insights}
\textit{Write down definitions, theorems, or takeaways. Use this space for concise notes.}
\subsubsection*{Definitions} 
\subsubsection*{Theorems}
\subsubsection*{Takeaways}

\subsection*{Problems Attempted}
\begin{enumerate}
    \item \textit{Problem statement or reference.}
    \item \textit{Solution (include partial work if needed).}
\end{enumerate}

\subsection*{Follow-Up Questions}
\begin{itemize}
    \item When is the Fourier Transform a smart thing to use? I remember Brian said something about that an instinctive response when seeing a function \(f(r_i - r_j)\) should be to Fourier Transform. How come? Are there other "forms" of functions where it immediately simplifies the problem (most of the time). What is \(k\)-space, and why does the Fourier Transform take us there? 
    
    Also, look at the picture you took of the blackboard last week when Jens Paaske wrote something about Fourier Transforming a single wave packet. Try it out for yourself with the normal distribution as the wave. Here you of course have to figure out how to write it "as a wave" (problably just replace \(x\) with \(x - vt\)) as well as how to integrate it properly. Exciting!
    \item How come the retarded potential formulation is not equivalent to putting a heaviside inside the integration.
    \item Show what Gauge invariance means
    \item Lorenz Gauge (\(L = 0\))
    \item Walk through the full retarded potential derivation.
    \item Rederive boundary conditions for the fields (both free and in matter)
    \item Learn more about the 4-vector formulation and how \(g_{\mu \nu}\) can "raise or lower indecies"
    \item Lorentz tansformationts as rotations in 4-space (Mogen's notes)
    \item Coloumb Gauge (\(\nabla \cdot \mathbf{A}\) )
    \item \textred{Walk through Mathemaniac derivation again. Read Dotson's resource from Arizona state. Do the Mathemaniac problem.}
    \item Why do we have to change \(V\) and \(\mathbf{A}\) simultaneously to have a proper Gauge transformation? Try to write out the potential formulation of Maxwell's equations and walk through the conditions imposed by the look of the Gauge Transformation. What happens if you only shift say \(V\). What about only shifting \(\mathbf{A}\)? 
\end{itemize}
