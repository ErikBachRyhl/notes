\section{Session Date: 6th of January, 2025}
\subsection*{Prestudy from yesterday}
\subsubsection*{To Read}
\begin{itemize}
    \item Chapter 11 in Griffiths
    \item Fourier Transform Chapter in Arfken probably. 
\end{itemize}
\subsubsection*{Problems}
\begin{itemize}
    \item Exam and a problem from ch. 11 Griffiths.
    \item Fourier Transform of a square wave
    \item Fourier Transform of a Gaussian.
\end{itemize}

\subsection*{Main Topic: \textit{What you are working on/torwards. What is the context?}}
Didn't get around to any of the things because I spent it on assignment and exams. But I'll get around to it! I can alwas get back to the ideas above. 

One really cool argument that Jörg told me about to day in lab was how we saw that the wave pulse \textit{shape} had changed when the wave got reflected. And as such, we could immediately conclude that the amount of dampening of the reflected wave has to be frequency dependent. Why? Because in frequency space (after af Fourier transform), we see that the shape of the wave pulse depends on the amplitudes of the frequencies used to build it. This means that if all frequencies are damped equally, we can pull out in front, Fourier transform back to the space domain and then have a scaled version of our originally shaped waveform. But since the \textit{shape} is different, that means that some frequencies were damped more than others - in other words, the dampening is frequency dependent!
