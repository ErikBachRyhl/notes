\section{Session Date: 14th of January, 2025}
\subsection*{Main Topic: Electromagnetism}
\subsection*{Topics Covered}
\begin{itemize}
    \item Retarded time
\end{itemize}

\subsection*{Key Insights}
\paragraph{Retarded Time} The retarded time is given by \begin{align*}
    t_r = t - \frac{\griffr[2]}{c} = t - \frac{\left| \mathbf{r} - \mathbf{r}^{\prime}\right| }{c}
\end{align*}
This formula says, that when considering the electromagnetic fields at a \textit{fixed} point in space, \(\mathbf{r}\), while possibly having time dependent sources \(\rho (\mathbf{r}^{\prime}, t)\) and \(\mathbf{J}(\mathbf{r}, t)\), then the relevant time to consider from the viewpoint of the \textit{fixed} point in space, \(\mathbf{r}\), \textbf{isn't} the current time at the source, \(t\), but rather the state of the system at the time when the electromagnetic news left the sources. Since electromagnetic news travel at the speed of light, this relevant time is exactly what the clock said \(\griffr[2] / c\) seconds ago; or put another way, this was at the time \(t_r\). 

And since the contents of the sources can't move faster than the speed of light, only one retarded point contributes to the potentials at any given moment. For a particle moving on a trajectory parameterised by \(t\), we don't want to figure out the current distance that the charge sees to the fixed field point, but rather what distance the fixed field point sees to the moving charge (which is a distance away, such that also information about its \textit{position} is delayed). We thus get an equation for \(\griffr[2]\): \begin{align*}
    \griffr[2] = \left| \mathbf{r} - \mathbf{w}(t_r) \right| = c (t - t_r)
\end{align*}   
which says that the seperation distance to use is the one which the field point "saw" at \(t_r\), not the one at \(t\). And this seperation distance itself depends on \(t_r\), since the particles position is parameterised by time!  
