\section{Session Date: 9th of January, 2025}
\subsection*{Prestudy from yesterday}
\subsubsection*{To Study}
\begin{itemize}
    \item Liénard-Wiechert Potentials Derivation (again!)
    \item Chapter 11 in Griffiths
    \item Lorentz Transformations
    \item Contravariant and Covariant
    \item Metric
\end{itemize}

\subsection*{Main Topic: Electromagnetism and Quantum Mechanics}
\subsection*{Topics Covered}
\begin{itemize}
    \item Liénard-Wiechert Potentials Derivation (again!) 
\end{itemize}
\paragraph{Liénard-Wiechert Potentials Derivation (again!)}
Starting from \begin{align*}
    V(\mathbf{r}, t) = \frac{1}{4\pi\epsilon_0} \int_{\mathcal{V}}  \frac{\delta ^3 (\mathbf{r}^{\prime} - \mathbf{w}(t_r))}{\left| \mathbf{r} - \mathbf{r}^{\prime}  \right| }d \tau ^{\prime} 
\end{align*}
we substitute \begin{align*}
    \mathbf{u} = \mathbf{r}^{\prime} - \mathbf{w}(t_r)
\end{align*}
such that the Jacobian becomes \begin{align*}
    J_{ij} &= \frac{\partial u_i}{\partial r^{\prime} _j} \\
    &=\delta _{ij} - \frac{\partial}{\partial r^{\prime} _j}\left( t - \frac{\left| \mathbf{r} - \mathbf{r}^{\prime}  \right| }{c} \right)  
\end{align*}
Then we can calculate, take the determinant again with the epsilon-version of the Leibniz-formula (which you see above) and use the fact that many of the terms disappear because they are of the form \(\epsilon _{ijk}T^{ij}\) where \(T^{ij}\) is symmetric (which we showed to always give zero yesterday).  

\textbf{Why isn't \(\mathbf{r}^{\prime} = \mathbf{w}(t_r)\)?} 

This is because the retarded time we are considering is a fixed vector \textit{for each infinitesimal volume element \(d \tau ^{\prime} \) in \(\mathcal{V}\)} which the integral is over. Thus the retarded time is not the retarded time relating to the point particles position (which itself depends on retarded time, giving us a quadratic like the example in Griffiths). In the integral, it is a fixed vector but \textit{for each \(d \tau ^{\prime} \)}, such that it isn't by any means a constant. The rest of the derivation is just math from here.

\paragraph{Lorentz Transformations} \begin{align*}
    &\overline{x} = \gamma (x - vt)\\
    &\overline{t} = \gamma (t - \frac{vx}{c^{2}})
\end{align*}

\paragraph{Covariant, Contravariant and Metric}
A covariant vector is written as \(a_\mu\) and a contravariant vector is written as \(a^{\mu}\). They differ only by the sign in the zeroth component (temporal) such that we may formally write \begin{align*}
    a_\mu = (-a_0, a_1, a_2, a_3) \equiv (a^0, a^1, a^2, a^3)
\end{align*} 
Defining the Minkowski metric as \begin{align*}
    g_{\mu \nu} = \begin{pmatrix}
        -1 & 0 & 0 & 0\\
        0 & 1 & 0 & 0\\
        0 & 0 & 1 & 0\\
        0 & 0 & 0 & 1
    \end{pmatrix}
\end{align*}
we see that \(g_{\mu  \nu }\) can raise and lower indicies \begin{align*}
    &g^{\mu \nu }a_\nu  = a^{\mu} \\
    & g_{\mu \nu } a^{\nu } = a_\mu 
\end{align*}
A greek letter always runs from 0 to 3 while a roman letter only runs from 1 to 3 (only spatial coordinates). We see that the interval \begin{align*}
    I \equiv (\Delta x)^{\mu}(\Delta x)_{\mu} = - (\Delta x^0)^{2} + (\Delta x^1)^{2} + (\Delta x^2)^{2} + (\Delta x^3)^{2} 
\end{align*}
gives the invariant spacetime interval!

\subsection*{Problems Attempted (Griffiths)}
\begin{enumerate}
    \item 12.46
    \item 12.47
    \item 12.48
    \item 12.49
\end{enumerate}

\subsection*{Follow-Up Questions/Ideas/ToDos}
\begin{itemize}
    \item Lorentz Transformations as a matrix-vector product (\(\Lambda _\mu ^\nu\)) 
    \item Metric invariant under similarity transform with the Lorentz Matrix
\end{itemize}
