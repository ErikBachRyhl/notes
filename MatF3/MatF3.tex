\documentclass[a4paper]{article}
\usepackage[a4paper, margin=1in]{geometry} % Adjust margin here, e.g., 1 inch
% Some basic packages
\usepackage[utf8]{inputenc}
\usepackage[T1]{fontenc}
\usepackage{textcomp}
\usepackage[english]{babel}
\usepackage{url}
\usepackage[demo]{graphicx}
\usepackage{float}
\usepackage{booktabs}
% \usepackage{enumitem}
\usepackage{enumerate}
\usepackage[colorlinks]{hyperref}

\pdfminorversion=7

% Don't indent paragraphs, leave some space between them
\usepackage{parskip}
\usepackage{changepage}

% Hide page number when page is empty
\usepackage{emptypage}
\usepackage{subcaption}
\usepackage{multicol}
\usepackage[dvipsnames]{xcolor}

% Other font I sometimes use.
% \usepackage{cmbright}

% Math stuff
\usepackage{amsmath, amsfonts, mathtools, amsthm, amssymb}

% Add this line to make equation numbering follow section
\numberwithin{equation}{section}

% Fancy script capitals
\usepackage{mathrsfs}
\usepackage{cancel}
% Bold math
\usepackage{bm}
% Some shortcuts
\newcommand\N{\ensuremath{\mathbb{N}}}
\newcommand\R{\ensuremath{\mathbb{R}}}
\newcommand\Z{\ensuremath{\mathbb{Z}}}
\renewcommand\O{\ensuremath{\emptyset}}
\newcommand\Q{\ensuremath{\mathbb{Q}}}
\newcommand\C{\ensuremath{\mathbb{C}}}

% Easily typeset systems of equations (French package)
\usepackage{systeme}

% Put x \to \infty below \lim
\let\svlim\lim\def\lim{\svlim\limits}

%Make implies and impliedby shorter
\let\implies\Rightarrow
\let\impliedby\Leftarrow
\let\iff\Leftrightarrow
% \let\epsilon\varepsilon

% COURSE SPECIFICS
% GRIFFITHS
\ifdefined\pdfliteral
    \let\griffPdfliteral\pdfliteral
\else \def\griffPdfliteral#1{\special{pdf: literal #1}} \fi

\newcommand\griffr[1][2]{\leavevmode\hbox{\kern1pt\vbox to1ex{}\griffPdfliteral{%
    q 1 J .27 0 0 .27 0 0 cm #1 w
    0 2 m
    0 2 8.1 9.7 9.2 13.2 c
    10.4 16.8 8.4 15.4 8 14.7 c
    7.6 14 6.8 12.6 12 13 c
    17 13.5 14.5 7.8 13.7 6 c
    12.8 4.3 10.3 1.2 11.4 .2 c
    12.6 -.7 18.8 3.6 18.8 3.6 c
    18.8 3.6 l S Q
}\kern6pt}}
\newcommand\hatgriffr{\skew3\hat{\griffr[4]}}

% Add \contra symbol to denote contradiction
\usepackage{stmaryrd} % for \lightning
\newcommand\contra{\scalebox{1.5}{$\lightning$}}

% \let\phi\varphi

% Command for short corrections
% Usage: 1+1=\correct{3}{2}

\definecolor{correct}{HTML}{009900}
\newcommand\correct[2]{\ensuremath{\:}{\color{red}{#1}}\ensuremath{\to }{\color{correct}{#2}}\ensuremath{\:}}
\newcommand\green[1]{{\color{correct}{#1}}}

% horizontal rule
\newcommand\hr{
    \noindent\rule[0.5ex]{\linewidth}{0.5pt}
}

% hide parts
\newcommand\hide[1]{}

% si unitx
\usepackage{siunitx}
\sisetup{locale = FR}

% Environments
\makeatother
% For box around Definition, Theorem, ...
% \usepackage{mdframed}
\usepackage[framemethod=TikZ]{mdframed}

% Custom command to draw a rectangular border around an equation
\setlength{\fboxsep}{5pt}  % Adjust padding inside the box
\usepackage{empheq}

\usepackage{environ}  % This package allows for easier custom environment definitions

% Define the custom environment
\NewEnviron{framed}{%
  \begin{empheq}[box=\fbox]{align}
  \BODY
  \end{empheq}
}
% Custom environment to box align equations
% \newenvironment{boxedalign}
%   {\begin{empheq}[box=\fbox]{align}}
%   {\end{align}\end{empheq}}

\newtheorem{thm}{Theorem}[subsection]
\newtheorem{defi}[thm]{Definition}
\newtheorem{lem}[thm]{Lemma}
\newtheorem{ret}{Correction}


\newtheorem*{term}{Terminology}
\newtheorem*{key}{Keywords and Related Concepts}
\newtheorem{lign}[thm]{Equation}
\newtheorem{law}[thm]{Law / Principle}

\usepackage{mathtools}
\DeclarePairedDelimiter\bra{\langle}{\rvert}
\DeclarePairedDelimiter\ket{\lvert}{\rangle}
\DeclarePairedDelimiterX\braket[2]{\langle}{\rangle}{#1\,\delimsize\vert\,\mathopen{}#2}


% \newcounter{theo}[section]
% \renewcommand{\thetheo}{\arabic{section}.\arabic{theo}}

% \mdfsetup{skipabove=1em,skipbelow=0em}
% \theoremstyle{definition}
% \newmdtheoremenv[nobreak=true]{definition}{Definition}
% \newmdtheoremenv[nobreak=true]{theorem}{Theorem}
% \newmdtheoremenv[nobreak=true]{corollary}{Corollary}
% \newmdtheoremenv[nobreak=true]{lemma}{Lemma}

% \newtheorem*{observation}{Observation}
% \newtheorem*{property}{Property}
% \newtheorem*{postulate}{Postulate}
% \newtheorem*{conclusion}{Conlusion}
% \newtheorem*{repitition}{Repitition}
% \newtheorem*{example}{Example}
% \newtheorem*{question}{Question}
% \newtheorem*{intuition}{Intuition}

% End example and intermezzo environments with a small diamond (just like proof
% environments end with a small square)
% \usepackage{etoolbox}
% \AtEndEnvironment{example}{\null\hfill$\diamond$}%
% \AtEndEnvironment{repitition}{\null\hfill$\diamond$}%
% \AtEndEnvironment{opmerking}{\null\hfill$\diamond$}%

% Fix some spacing
% http://tex.stackexchange.com/questions/22119/how-can-i-change-the-spacing-before-theorems-with-amsthm
\makeatletter
\def\thm@space@setup{%
  \thm@preskip=\parskip \thm@postskip=0pt
}


% Exercise 
% Usage:
% \oefening{5}
% \suboefening{1}
% \suboefening{2}
% \suboefening{3}
% gives
% Oefening 5
%   Oefening 5.1
%   Oefening 5.2
%   Oefening 5.3
\newcommand{\exercise}[1]{%
    \def\@exercise{#1}%
    \subsection*{Exercise #1}
}

\newcommand{\subexercise}[1]{%
    \subsubsection*{Exercise \@exercise.#1}
}

\usepackage{xcolor}
\newcommand{\textred}[1]{\textcolor{red}{#1}}

% \lecture starts a new lecture (les in dutch)
%
% Usage:
% \lecture{1}{di 12 feb 2019 16:00}{Inleiding}
%
% This adds a section heading with the number / title of the lecture and a
% margin paragraph with the date.

% I use \dateparts here to hide the year (2019). This way, I can easily parse
% the date of each lecture unambiguously while still having a human-friendly
% short format printed to the pdf.

\usepackage{xifthen}
\def\testdateparts#1{\dateparts#1\relax}
\def\dateparts#1 #2 #3 #4 #5\relax{
    \marginpar{\small\textsf{\mbox{#1 #2 #3 #5}}}
}

\def\@lecture{}%
\newcommand{\lecture}[3]{
    \ifthenelse{\isempty{#3}}{%
        \def\@lecture{Lecture #1}%
    }{%
        \def\@lecture{Lecture #1: #3}%
    }%
    \subsection*{\@lecture}
    \marginpar{\small\textsf{\mbox{#2}}}
}

\def\@chapter{}%
\newcommand{\chapter}[3]{
    \ifthenelse{\isempty{#3}}{%
        \def\@chapter{Chapter #1}%
    }{%
        \def\@chapter{Chapter #1: #3}%
    }%
    \subsection*{\@chapter}
    \marginpar{\small\textsf{\mbox{#2}}}
}

\def\@week{}%
\newcommand{\week}[3]{
    \ifthenelse{\isempty{#3}}{%
        \def\@week{Uge #1}%
    }{%
        \def\@week{Uge #1: #3}%
    }%
    \subsection*{\@week}
    \marginpar{\small\textsf{\mbox{#2}}}
}

% These are the fancy headers
% \usepackage{fancyhdr}
% \pagestyle{fancy}

% LE: left even
% RO: right odd
% CE, CO: center even, center odd
% My name for when I print my lecture notes to use for an open book exam.
% \fancyhead[LE,RO]{Gilles Castel}

% \setlength{\headheight}{5pt}

% % \fancyhead[R]{\@lecture} % Right odd,  Left even
% \fancyfoot[R]{\thepage}  % Right odd,  Left even
% \fancyfoot[C]{\leftmark}     % Center

\makeatother

% Todonotes and inline notes in fancy boxes
\usepackage{todonotes}
\usepackage{tcolorbox}

% Make boxes breakable
\tcbuselibrary{breakable}

% Usage: 
% \begin{correction}
%     Lorem ipsum dolor sit amet, consetetur sadipscing elitr, sed diam nonumy eirmod
%     tempor invidunt ut labore et dolore magna aliquyam erat, sed diam voluptua. At
%     vero eos et accusam et justo duo dolores et ea rebum. Stet clita kasd gubergren,
%     no sea takimata sanctus est Lorem ipsum dolor sit amet.
% \end{correction}
\newenvironment{correction}{\begin{tcolorbox}[
    arc=0mm,
    colback=white,
    colframe=green!60!black,
    title=Correction,
    fonttitle=\sffamily,
    breakable
]}{\end{tcolorbox}}

% Same as 'correction' but color of box is different
\newenvironment{note}{\begin{tcolorbox}[
    arc=0mm,
    colback=white,
    colframe=white!60!black,
    title=Note,
    fonttitle=\sffamily,
    breakable
]}{\end{tcolorbox}}


% Figure support as explained in my blog post.
\usepackage{import}
\usepackage{xifthen}
\usepackage{pdfpages}
\usepackage{transparent}
\newcommand{\incfig}[1]{%
    \def\svgwidth{\columnwidth}
    \import{./figures/}{#1.pdf_tex}
}

% Fix some stuff
% %http://tex.stackexchange.com/questions/76273/multiple-pdfs-with-page-group-included-in-a-single-page-warning
\pdfsuppresswarningpagegroup=1


% My name
\author{Erik Bach Ryhl}


\usepackage{slashed}

\graphicspath{ {./figs/} }

\setcounter{tocdepth}{4}
\setcounter{secnumdepth}{3}


\theoremstyle{definition} % Define theorem styles here based on the definition style (used for definitions and examples)
\newtheorem*{definition}{Definition}

\theoremstyle{plain} % Define theorem styles here based on the plain style (used for theorems, lemmas, propositions)
\newtheorem{theorem}{Theorem}[section]
\newtheorem{corollary}[theorem]{Corollary}
\newtheorem{lemma}[theorem]{Lemma}
\newtheorem{proposition}[theorem]{Proposition}
\newtheorem*{problem}{Problem}

\theoremstyle{remark} % Define theorem styles here based on the remark style (used for remarks and notes)
\newtheorem{example}[theorem]{Example}
\newtheorem*{notation}{Notation}
\newtheorem{remark}[theorem]{Remark}
\newtheorem*{solution}{Solution}



\title{Group Theory in a Nutshell for Physicists}
\begin{document}
    \maketitle
    \tableofcontents
    \newpage
    \section{Notes to self}
\subsection{To Add}
\begin{itemize}
    \item An explanation of why \begin{align*}
        \epsilon^{ijk \cdots n} R^{ip} R^{jq} R^{kr} \cdots R^{ns} = \epsilon^{p q r \cdots s} \mathrm{det}\ R  
    \end{align*}
    \item The more formal definition of a representation
\end{itemize}
\subsection{From Ludvig}
The elements of $SO(3)$ are $3D$ rotations, so the generators represented by $J_x, J_y, J_z$ generate infinitesimal rotations. Using Noether's theorem, the angular momentum operator (about some axis) in quantum mechanics is defined as the generator of infinitesimal rotations (about that axis). This is exactly the same as the Lie group! The generators correspond to operators, the exponentiated group elements correspond to actual rotations of particle. Thus, the $J_z$ angular momentum operator is (conceptually) the same as the $J_z$ in the Lie algebra of $SO(3)$. Zee writes it as $\vec{L}_{\text{operator}} = \hbar \vec{L}_{\text{Lie}}.$

Now, let $J_z \ket{m} = m \ket{m}$. Here, the $\ket{m}$ kets are the normalised eigenvectors of $J_z$, meaning they span the "eiggenspace" of the $J_z$ matrix. This definition is the same as a tensor furnishing a representation of $J_z$ in group theory; it is something, the representation of $J_z$ can be applied to by matrix multiplication from the left. Thus, tensors furnishing representations of $J_z$ are directly connected to the eigenfunctions $\ket{m}$. This means that the number of eigenvalues $m$ is equal to the dimension of the corresponding representations. Interlude:

\textit{A tensor representation of $SO(3)$ can be similarity transformad into a block-diagonal form, where the irreducible representations are furnished by traceless, symmetric tensors. They have dimensions equal to the number of independent indices of these, which for $j$ components are $2j+1$.}

Now, as the $\ket{m}$ kets transform among themselves under rotations, i.e. by $J_z$ operating on them (that's the definition of an eigenket), the corresponding representations must do the same. Therefore, the eigenkets and eigenvalues correspond to tensors furnishing \textit{irreducible} representations of the corresponding operator/representation matrix. 

For a given $j$, there are $2j+1$ corresponding $\ket{m}$ kets. Thus, if $j$ is integer, these correspond to the irreducible tensor representations of $SO(3)$ before exponentiation, or the irreducible representations of the Lie algebras. 

Now for angular momentum addition. By considering infinitesimal rotations, it is shown in Zee that $J_z \ket{j, m} \otimes \ket{j', m'} = (m+m')$. This leads to the conclusion given at the last lecture: “Taking direct products of irreducible representations of SO(3) \& SO(3) and decomposing into irreducible representations = adding angular momentum and possibly spin in QM”.

Another thing is that the Clebsch-Gordan decomposition of tensor representations yields the same possible values as the the possible values of $m$ are in quantum mechanics:
\begin{align}
	j \otimes j' = (j+j') \oplus (j+j'-1) \oplus \dots \oplus (j-j'), \quad \text{for } j > j'.
\end{align}

    % \section{Lecture 1}
    % \subsection{Linear Algebra Review}
Consider the linear system of equations \begin{align*}
    P \vec{x} = \vec{u}
\end{align*}
where \(P\) and \(\vec{u}\) is known. It can be solved formally by multyplying with the inverse matrix on both sides \begin{align*}
    \vec{x} = P^{-1} \vec{u}
\end{align*}

\begin{theorem}[Determinant]
    The entries in the inverse of a matrix \(M\) are ratios of a number called the \textbf{determinant}, which for an \(n\times n\) matrix can be computed explicitly as follows: \begin{align*}
        \boxed{\mathcal{D} = \sum_{P \in S_n} \mathrm{sgn}(P) \prod_{i = 1}^{n} M_{i, P(i)}} \tag{56}
    \end{align*}
\end{theorem}

\begin{remark}
    The determinant can also be written as (Einstein Summation implied) \begin{align*}
        \det M = \epsilon ^{i_1 i_2 \cdots i_n}  M^{1i_1} M^{2i_2} \cdots M^{ni_n}
    \end{align*}
    Even more generally, it is written in the book that \begin{align*}
        \epsilon ^{pqr \cdots s} \det M = \epsilon ^{i j k \cdot m} M^{ip}M^{jq} M^{kr} \cdots M^{ms} \tag{67}
    \end{align*}
    which expresses the tensorial transformation property of the Levi-Civita symbol. I don't know what this means yet, but hopefully we will in a few weeks.
\end{remark}

\begin{theorem}[Cramer's Formula for the Inverse]
    The inverse of an \(n\times n\) matrix is given by \begin{align*}
        \left( M^{-1}  \right)_{ij} = \frac{1}{\mathcal{D} } (-1)^{i + j} \det \tilde{M}(\slashed{j}, \slashed{i}) \tag{67}
    \end{align*}
\end{theorem}

\textbf{Transpose}\begin{align*}
    \left( A \right)_{ij} ^T & := A_{ji}\\  
    \left( AB \right) ^T &= B^T A^T
\end{align*} 
\textbf{Trace}
\begin{align*}
    &\mathrm{tr}\ M := \sum_i M_{ii}\\
    &\mathrm{tr}\ AB = \mathrm{tr}\ BA
\end{align*} 
Note that the above equality implies that trace is left invariant under cyclic permutations of the matrices involved.

\subsubsection{Properties of the Determinant}
\begin{enumerate}
    \item Multiplying a row (column) of the matrix \(M\) with \(\lambda\) scales the determinant by the same factor, \(\mathcal{D} \to \lambda \mathcal{D} \).
    \item Interchanging rows (columns) flips the sign of the determinant.
    \item Repeated rows (columns) gives a determinant of zero.
    \item Subtracting scaled rows (columns) from other rows (columns) leaves the determinant unchanged.
    \item \(\det M = \det M^T\) 
\end{enumerate}
These properties ensure that any matrix can be transformed to an upper triangular matrix without changing the determinant. Once a matrix is upper triangular, the determinant is just the product of the diagonal (as seen by Laplace expansion).

If \begin{align*}
    B \vec{v} = 0
\end{align*}
then a solution can only exist if \(\det B = 0\) as well.

\subsubsection{Eigenvectors and Eigenvalues}
\begin{align*}
    \det (M - \lambda I) = 0
\end{align*}
gives an \(n\)-degree polynomium which implies \(n\) solutions. This means that one can use eigenvectors as a basis. 

\subsubsection{Properties of Hermitian Matrices}
\begin{definition}[Hermitian Matrices]
    The Hermitian conjugate of a matrix is defined as follows:
\begin{align*}
    M^{\dagger} \coloneqq \left( M^{\ast}  \right)^T
\end{align*}
A matrix is said to be Hermitian if \(M^{\dagger} = M\) 
\end{definition}

A Hermitian Matrix \begin{enumerate}
    \item Has only real eigenvalues
    \item Orthogonal eigenvectors spanning the space. These can be normalized. Thus:
\end{enumerate}
\begin{align*}
    H = H^{\dagger} \implies \lambda _a = \lambda _a ^{\ast} \text{ and } \psi_a ^{\dagger} \psi _b = \delta _{ab}
\end{align*}


\subsubsection{Upper and Lower Indicies}
Since \begin{align*}
    \phi^{\dagger} \psi = \left( \phi ^{\ast}  \right) ^T \psi = \sum_{i} \phi _i ^{\ast} \psi _i 
\end{align*} 
we now change our notation such that complex vectors carry an upper index, \(\psi ^i\). And for each such complex vector, we introduce a vector carrying a lower index, where one raises or lowers an index through complex conjugation. As such, \begin{align*}
    \psi _i \equiv \left( \psi ^i \right) ^{\ast} 
\end{align*} 
With this definition, we get the notation \begin{align*}
    \phi ^{\dagger} \psi = \phi _i \psi ^i = \sum_{i} \phi ^{i \ast} \psi ^i  
\end{align*}
We now introduce a rule, which will be able to help us out like unit analysis: \begin{notation}
    An upper (lower) index always needs to be contracted with a lower (upper) index. 
\end{notation}

This also means that we will write the \((i, j)\)-th element of the matrix \(M\) as \(M^i _j \) such that \begin{align*}
    \left( M \psi  \right) ^i = M^i_j \psi ^j 
\end{align*}  
In that sense, an upper index is a row index and the lower index is a column index.

\subsubsection{Diagonalization}
Letting \(S = \left( \psi _1, \psi _2, \dots , \psi _n \right) \) be the matrix with the \(n\) eigenvectors of the matrix \(M\) as it's columns. Then the \textit{similarity transformation} of \(M\) with \(S\) will result in a diagonal matrix with the eigenvalues along the diagonal. As an equation: \begin{align*}
    S^{-1} M S = \mathrm{diag} \left\{ \lambda _1, \lambda _2, \dots, \lambda _n \right\} \coloneqq  \Lambda 
\end{align*} 

\begin{definition}[Unitary and Orthogonal Matrices]
    A unitary matrix is any matrix satisfying \begin{align*}
        U ^{\dagger} U = I.
    \end{align*}
    An orthogonal matrix is any matrix satisfying \begin{align*}
        O^T O = I.
    \end{align*}
    Thus a real unitary matrix is orthogonal.
\end{definition}

This gives us the following "theorem" (basically the definition of unitary matrices, actually): 
\begin{theorem}[Diagonalization of Hermitian Matrices]
    A Hermitian matrix can, by definition, be diagonalized into a purely real matrix by similarity transform with a unitary matrix \begin{align*}
        U ^{\dagger} M U = \Lambda 
    \end{align*}
\end{theorem}

\begin{theorem}[Trace equals sum of Eigenvalues]
    The trace of a matrix is the sum of its eigenvalues.
\end{theorem}
\textit{Proof:} We have already seen that taking the trace is cyclic. Thus \begin{align*}
    \mathrm{tr}\ S^{-1} M S = \mathrm{tr}\ S^{-1} S M = \mathrm{tr}\ M = \mathrm{tr}\ \Lambda = \sum_{a} \lambda _a 
\end{align*}

\begin{theorem}[Simultaneous Diagonalization]
    If two matrices commute, they will be diagonalized by the same similarity transformation.
\end{theorem}

\subsubsection{Functions of Matrices}

\subsection{Why Group Theory?}
Group Theory is the mathematics of symmetry. But what exactly is meant by symmetry? \begin{definition}
    A \textit{symmetry} is an invariance under a transformation.
\end{definition}
But what then is a transformation?
\begin{definition}
    A \textit{transformation} is any operation satisfying the following: \begin{itemize}
        \item Composition
        \item Invertability
    \end{itemize}
\end{definition}

In classical physics, a symmetry \(\implies \) conserved quantity (Nöether's Theorem). For example, invariance under a translation means that momentum is conserved and if there is an invariance under a rotation, then the angular momentum is conserved.

In quantum physics, we have that if there is a symmetry under the action of an operator, then \begin{align*}
    \textit{Symmetry} \iff  \left[ \hat{O}, H \right] = 0
\end{align*}
In other words, the operator responsible for this symmetry commutes with the Hamiltonian operator. this implies that the Hamiltonian operator and the \(\hat{O}\) have the same eigenvalues and eigenvectors. And these eigenvalues are in some sense weights of the group while the eigenvectors form a representation of the group. We'll get to know what all of this means!

There are mainly two kinds of symmetries met in physics: \begin{enumerate}[i)]
    \item Space-time 
    \item Internal
\end{enumerate}

\textbf{Examples of space-time symmetries} \begin{itemize}
    \item Translation
    \item Rotation
    \item Lorentz Boost
\end{itemize} 
If one has rotation + boost symmetry, then one has "Lorentz symmetry". If one includes translation, then one has "Poincaré symmetry". Further examples include \begin{itemize}
    \item Dilatation: \(X^{\mu} \to  \lambda  X^\mu \)
    \item Inversion: \(X^\mu \to  \frac{X^\mu}{\left| X \right| ^{2} }\) (of the unit sphere)
\end{itemize}

All the above taken together are known as "Conformal Symmetries" (angle preserving). This is the basis for Conformal Field Theories (CFT). One can also introduce symmetry under "supertranslations" and "superboosts". Then we can form a superconformal field theory (SCFT).

\textbf{Examples of internal symmetries}
\begin{itemize}
    \item Permutation symmetry
    \item Gauge symmetry
\end{itemize} 

In electromagnetism, the gauge symmetry is described by the group \(U(1)\) (invariance under phase shifts). In the standard model, one meets the gauge symmetry \(SU(3) \times SU(2) \times U(1)\). In heterotic string theory, the group is \(SO(32)\), or \(E_8 \times E_8\).

A new field of research which has gained recent traction is known as "non-invertible symmetries". One cannot describe this by use of groups, but by "Fusion Categories" instead. Maybe this will be hot in the future. 

\newpage
\Large{\textbf{Course Structure}}
\normalsize
\begin{itemize}
    \item Basic definitions and examples
    \item Representation Theory \begin{itemize}
        \item Character Tables
        \item Tensor Representation of \(SO(N)\) (especially \(SO(3), SO(4)\) )
        \item Tensor Repr. of \(SU(N)\)
        \item Spinor Repr. of \(SO(N)\) (fermions) 
    \end{itemize}
    \item Lie Groups \begin{itemize}
        \item Roots, Weights
        \item Dynkin Diagrams
        \item Classification of Lie Groups (Cartan's Classification)
    \end{itemize}
\end{itemize}
\newpage
\subsection{Introduction}
\begin{definition}
    A group \(\left( \mathcal{G} , \circ \right) \)  is a set of elements \(\mathcal{G} = \left\{ g_{\alpha }  \right\} \) (cont. or discrete) together with a group operation \(( \circ)\) satisfying the following axioms \begin{enumerate}[i)]
        \item \(g_{\alpha }, g_{\beta } \in \mathcal{G}   : g_{\alpha } \circ g_{\beta } \in \mathcal{G}   \) (closure)
        \item \(\left( g_{\alpha } \circ g_{\beta }   \right) \circ g_{\gamma } = g_{\alpha } \circ  \left( g_{\beta } \circ g_{\gamma }   \right)   \) (associativity)
        \item \(\exists I: g_{\alpha } \circ  I = I \circ  g_{\alpha } = g_{\alpha }   \) (existence of identity)
        \item \(\forall g_{\alpha } \in \mathcal{G}  : \exists g_{\alpha } ^{-1} : g_{\alpha } \circ  g_{\alpha } ^{-1} = g_{\alpha } ^{-1} \circ g_{\alpha } = I  \) (existence of inverse)
    \end{enumerate} 
\end{definition}

\begin{definition}
    If the group operation is commutative, the group is said to be \textbf{Abelian}. 
\end{definition}

\underline{Space time examples: }\((\mathbb{Z} , +)\) is Abelian, whereas \((\mathbb{R} ^3, O(3))\) is non-Abelian. "Boosts" are also commutative.

\underline{Internal examples:} \(SL(N, \mathbb{R} ) = (\left\{ N\times N \text{ real matrices with det}(M) = 1  \right\}, \circ )\)  

\subsubsection{Finite Groups and Multiplication Tables}
\begin{definition}
    Let \(\mathcal{G}\) be a group with a finite set of elements \(\left\{ g_1, \dots , g_n \right\} \). We then define the \textbf{order} of the group to be the cardinality of the set of group members: \begin{align*}
        \rm{order}(\mathcal{G} ) \coloneqq \left| \left\{ g_{\alpha } \right\}  \right|  = n 
    \end{align*} 
\end{definition}

For any finite group of order \(n\) one can construct a \(n \times n\) \textit{multiplication table} where entry \((i, j)\) is given by \(g_i \circ g_j\). 

\begin{theorem}["Sudoku Theorem"]
    All group elements of a finite group \(\mathcal{G}\) appear once, and only once, in each row and in each column.  
\end{theorem}
\textit{Proof:} Let \(g_i, g_j, g_k \in \mathcal{G}\). Assume \(j \neq k\). We see that if the same element \(g_i\) multiplies an element both in column \(j\) and in column \(k\) of the multiplication table, then  
\begin{align*}
    g_i \circ  g_j = g_i \circ  g_k \implies  g_i ^{-1}\circ  g_i \circ  g_j = g_i ^{-1} \circ  g_i \circ g_k \implies g_j = g_k \implies j = k\quad \contra
\end{align*} 
Thus we conclude that \(j = k\); we could equally well have gone through the proof with the rows, since the we can, by our axioms, multiply the inverse both from the right and from the left to get the identity element of the group.

\hfill \qed 

\begin{remark}
    The consequence of the above theorem is, that if one one can make a multiplication table that \textit{has} a repeated element in a row or a column, then the elements used to form that multiplication table do not constitute a group. We also see that Abelian groups necessarily have symmetric multiplication tables. Note also that since the group elements forms an unordered set (all members are unique), then the last implication above is unnecessary to write, since the prior equality is already stating that \(j = k\); the fact that they are "tacked onto" a \(g\) is just to remind ourselves that we are working with a group.   
\end{remark}

\subsubsection{Group Isomorphisms}
\begin{definition}
    Let \(\mathcal{G} \) and \(\mathcal{G} ^{\prime} \) be groups. Let \(f : \mathcal{G} \to  \mathcal{G} ^{\prime} \) be a map between them. We say that the map is a \textbf{homomorphism} between the groups if \begin{align*}
        f(g_1 \circ g_2) = f(g_1) \circ f(g_2)
    \end{align*}
    If the map \(f\) is bijective, we say that the map is an \textbf{isomporhism} between the groups, and that the groups are group-theoretically \textbf{isomorphic}: \begin{align*}
        \mathcal{G} \cong_{\text{grp}} \mathcal{G}^{\prime}  
    \end{align*}   
    The map \(f\) provides a \textbf{structure-preserving, one-to-one mapping} between the groups. 
\end{definition}

\begin{proposition}
    Let \(f : \mathcal{G} \to \mathcal{G} ^{\prime} \) be an isomorphism. Then \begin{enumerate}[i)]
        \item \(f(I_{\mathcal{G} } ) = I_{\mathcal{G} ^{\prime} }\)
        \item \(\left( f(g) \right) ^{-1} = f(g^{-1}) \)  
    \end{enumerate} 
\end{proposition}
\textit{Proof}: Let \(g^{\prime} \in \mathcal{G}^{\prime} \implies \exists g : f(g) = g^{\prime}\)  
\begin{enumerate}[i)]
    \item and since \(f(g) = f(I_{\mathcal{G} } \circ g ) = f(I_{\mathcal{G} } ) \circ f(g) = f(I_{\mathcal{G} }) \circ g^{\prime}\) we find that \(g^{\prime} = f(I_{\mathcal{G}} ) \circ g^{\prime}\implies f(I_{\mathcal{G} }) = I_{\mathcal{G}^{\prime} } \).
    \item and then we have that \(I_{\mathcal{G} ^{\prime}} = f(I_{\mathcal{G} } ) = f(g \circ g^{-1}) = f(g) \circ f(g^{-1}) \implies \left( f(g) \right)^{-1} = f(g^{-1})\).
\end{enumerate}
\hfill\(\qed\) 

\begin{theorem}
    Let \(p\) be prime. Then there is only one group with \(p\) elements, which is \begin{align*}
        \mathbb{Z} _p \coloneqq \left( \left\{ 0, 1, \dots, p \right\}, +\ \rm{mod }\ p  \right) 
    \end{align*}  
\end{theorem}
We say that there "is only one group" in the sense that all other groups with \(p\) elements will be isomorphic to \(\mathbb{Z} _p\).  

\subsubsection{Construction of Groups}
There are several way to construct groups: \begin{itemize}
    \item Construct multiplication tables
    \item Use the Cartesian Product: Let \((F, \cdot ), (G, \circ)\) be groups. Then we see that \begin{align*}
        \left( F \times G, *\right) = \left( \left\{ (f, g)\ \big| \ f \in F, g\in G \right\}, \left( f_1, g_1 \right) * \left( f_2, g_2 \right) = \left( f_1 \cdot f_2, g_1 \circ g_2\right)   \right)
    \end{align*}
    also constitutes a group.
\end{itemize}

\begin{theorem}[Reformulation of Sylow's Theorems (informal)]
    There exists as many different Abelian groups of order \(n\) (with \(n\) elements) as there are "ways of writing" the number \(n\) as a factorisation of primes raised to an integer power.
\end{theorem}
\underline{Example:} As an example to illustrate what is meant by "ways of writing", consider \(n = 8\). With this theorem, we conclude that there exists 3 distinct, finite, abelian groups with 8 elements since we can write \begin{align*}
    8 &= 2^3\\
    &= 2^2 \cdot 2\\
    &= 2 \cdot 2 \cdot 2
\end{align*} 
Notice that \(2^2 \cdot 2\) is regarded as equivalent to \(2 \cdot 2^2\) in this sense of "different ways".  
    \newpage

    \section{Tensors and Representations of the Rotation Group \(SO(N)\)}
\begin{definition}[Representation (take 1)]
    A representation is homomorphic map \(D^{(n)} : \mathcal{G} \to GL (n, \mathbb{C})\). That is, a map from the group \(\mathcal{G}\) to the set of all linear transformations on a vector space \(V\) with \(\mathrm{dim} (V) = n\). That the map is homomorphic means that it \textit{preserves structure}: \begin{align*}
        D(g_1) D(g_2) = D(g_1 g_2)
    \end{align*}
    This means that any product between the representations of group elements (in the vector space) is equal to the representation of the products of the group elements (in the group).
\end{definition}

\begin{definition}[Direct Product]
    Given two mathematical objects \(A\) and \(B\) carrying any number of free indicies, we define the direct product between them by defining how to obtain its components:
    \begin{align*}
        (A \otimes B)_{} = A_{i, j, k, \cdots n} \otimes B_{p, q, r, \cdots s} \coloneqq 
    \end{align*}
\end{definition}

\subsection{The Special Unitary Groups \(SU(N)\)}
\begin{definition}[\(SU(N)\)]
    \(SU(N)\) are all the \(N \times N\) matrices with complex entries satisfying that \begin{align*}
        U ^{\dagger} U = 1, \quad \mathrm{det}\ U = 1
    \end{align*}
\end{definition}
The defining representation is defined by being furnished by complex vectors with \(N\) entires. Unitary matrices preserve inner products of the type \(v ^{\dagger} w\), since 
\begin{align*}
    v^{\prime \dagger} w^{\prime} = \left( U v \right) ^{\dagger} \left( U w \right) = v ^{\dagger} U ^{\dagger} U w = v^{\dagger} w
\end{align*} 
Also, we see that \begin{align*}
    1 = \mathrm{det} U ^{\dagger} U = \mathrm{det} U ^{\dagger} \mathrm{det} U = \mathrm{det} U ^{\ast} \mathrm{det} U = \left| \mathrm{det} U \right| ^{2} 
\end{align*}
which implies that \(\mathrm{det} U = e^{i \theta }\). We choose by convention that \(\mathrm{det}  U = 1\) as defined above. 

It turns out that \(SU(2)\) describes both fermions (spin) as well as weak interactinos (isospin). It is also the case that \(SU(3)\) describes QCD (quark interactions). The really cool idea is that (with coupling constants), the standard model is described by \begin{align*}
    SU(3) \otimes SU(2) \otimes U(1)
\end{align*}

Since \begin{align*}
    SU(3) \otimes SU(2) \otimes U(1) \subset SU(5)
\end{align*}
an attempt was made at a "Great Unified Theory" based on the symmetries of this group, but it hasn't worked out (and probably won't). 

One can also do expansions in \(SU(N)\) like:
\begin{align*}
    \left\langle O \right\rangle = O_0 + \frac{1}{N^{2} } O_1 + \frac{1}{N_4} O_2 + \cdots 
\end{align*}

\subsubsection{\(SU(N)\) as a Lie Group with Lie Algebra (\(su(N)\))}
To be a Lie Group, we should be able to expand our group element as an exponential map around the identity\begin{align*}
    g = \exp \left( i \sum_{a} \theta  ^a  T^ a\right) 
\end{align*}
where \(\theta ^a\) are the infinitesimal parameters, and \(T^a\) are the generators.

The Lie Bracket of the Algebra is a bilinear, antisymmetric product which needs to satisfy that \begin{align*}
    \left[ T^a, T^b \right] = i f^{abc} T^c
\end{align*}
as well as the Jacobi Identity.

Taking an element in our group, \(U = g = e^{i \epsilon h}\), we find that \begin{align*}
    1 = U^{\dagger} U = \left( 1 - i \epsilon h ^{\dagger} + O(\epsilon ^{2} ) \right)\left( 1 + i \epsilon h ^{\dagger} + O(\epsilon ^{2} ) \right) = 1 + i \epsilon (h - h ^{\dagger} + O(\epsilon ^{2} ))
\end{align*} 
By equating terms of like order, we see that our generators need to be Hermitian\begin{align*}
    h = h^{\dagger} 
\end{align*}

We see that \begin{align*}
    \det(U) = 
    \begin{vmatrix}
    1 + i \epsilon h_{11} &  i \epsilon h_{12}  & \cdots & i \epsilon h_{1n} \\
    i \epsilon h_{21} & 1 + i \epsilon h_{22}  & \cdots &  \\
    \vdots & \vdots & \ddots & \vdots \\
     &   & \cdots & 1 + i \epsilon h_{nn}
    \end{vmatrix}
\end{align*}
Keeping only the terms of first order in \(\epsilon\), we obtain \begin{align*}
    \det(U) = 1 + i \epsilon Tr(h) + O(\epsilon ^{2} )
\end{align*} 
which forces \(Tr(h) = 0\) 
\textred{Review this!}

Counting the number of generators in \(SU(N)\), we have \(N^{2} \) complex and \(N^{2} \) real components. But \(h ^{\dagger}  = h\) enforces a constraint, and the \(Tr(h) = 0\) also removes a single free component. Thus we have \begin{align*}
    N^{2} -1
\end{align*}
generators in \(SU(N)\).

\begin{example}
    \(su(3)\) has 8 generators written in terms of the Gell-mann Matrices \(\lambda ^1, \lambda ^2, \dots, \lambda ^8\). \begin{align*}
        T^a = \frac{\lambda ^a}{2}
    \end{align*}
    while we have \begin{align*}
        \left[ T^a, T^b \right] = i f^{abc} T^c
    \end{align*}
\end{example}

\subsection{Tensors for \(SU(N)\)}
In the defining repr., we have \begin{align*}
    v ^i \to (v^{\prime} )^i = U^i _j v^j
\end{align*}
Now, we find that the tensor representation will look like \begin{align*}
    T^{ij} \to \left( T ^{\prime}  \right) ^{i j} = U ^i _k U^j _l T^{kl}
\end{align*}

Or, higher rank: \begin{align*}
    T^{i_1 \dots i_n} \to \left( T^{\prime}  \right) ^{i_1 \dots i_n} = U^{i_1} _ {j_1} \cdots U^{i_n} _ {j_n} T^{j_1 \dots j_n}
\end{align*}
If we can find non-trivial subspaces in the product representation (which is furnished by these tensors), then the tensor repr. is reducible. This is equivalent to our tensors having specific symmetry properties. We find that \begin{align*}
    T^{ii} \to \left( T^{\prime}  \right) ^{ii} = U^i _k U^i _l T^{k l} = \left( U^T \right) ^k _i \left( U \right) ^i _ l T^{kl} 
\end{align*}
Thus unlike \(SO(N)\), the trace doesn't furnish an irreducible representation in \(SU(N)\), since \(U^T U \neq 1\) . Any two-rank tensor can thus be decomposed into two parts (constrast to three):
\begin{align*}
    T^{ij} = \frac{1}{2}\left( T^{ij} + T^{ji}  \right) + \frac{1}{2} \left( T^{ij} - T^{ji}   \right) 
\end{align*}

\subsection{The Conjugate Representation}
From a repr. \(D(g)\) we know that \(D^{\ast} (g)\) is a repr. as well (the conjugated repr.).

Now, consider \begin{align*}
 \left( v^i \right) ^{\ast} \to \left( U ^i _j \right)^{\ast} \left( v^j \right)^{\ast} = \left( U^{\dagger}  \right) ^j _i  \left( v^j \right) ^{\ast} 
\end{align*}
We now define an object that carries the conjugated repr. as an object that transforms like this \begin{align*}
    w_i \to \left( U^{\dagger}  \right) ^j _i w_j = w_j \left( U ^{\dagger}  \right) ^j _i
\end{align*}
where we use the lower index to denote these types of objects. The last equality is a reordering (these are components) to remind us that we want the matching indicies (upper and lower). We see that the \textit{components} of the vectors in the adjoint representation transforms like the basis vectors themselves in the defining representation. 

In \(SO(N)\), our components transform like \begin{align*}
    v^i \to R^i _j v^j
\end{align*}
whereas the basis vectors transform like \begin{align*}
    e^i \to R^j _ i v^j
\end{align*}

In the adjoint repr., the components transform like the basis vectors in \(SO(N)\)!

We then get \begin{align*}
    T_{i_1 \dots i_ n} \to T^{\prime} _{i_1 \dots i_n} = T_{j_1 \dots j_n} U^{\dagger j_1}_{i_1} \cdots U^{\dagger j_n}_{i_n} 
\end{align*}

And we can also combine!\begin{align*}
    T^{j_1 j_2}_{i_1 i_2} \to U^{j_1}_{l_1} U^{j_2} _{l_2} T^{l_1 l_2} _{k_1 k_2} U^{\dagger k_1}_{i_{1} } U^{\dagger k_2}_{i_2}  
\end{align*}

We can now take a generalized trace, which \textit{will} be invariant under an \(SU(N)\) transformation: \begin{align*}
    T^i _ i \to \left( T^{\prime}  \right) ^i _i = U^i _k T ^k _ l U^{\dagger l }_ i = \left( U^{\dagger} \right)^l _ i U ^i _k T^k _l = \delta_{kl} T ^k _ l = T^k _ k  
\end{align*}
An invariant traceeeeeee!

We can thus decompose \begin{align*}
    T^i _j = \underbrace{T^i _j - \frac{1}{N}\delta^i _ j T^k _k}_{irreducible} + \frac{1}{N}\delta ^i _ j T ^k _ k 
\end{align*}

Irreducible tensors: Specific symmetries in upper and lower indicies seperately. And they are traceless.

\begin{remark}
    One might think: "Where's the symmetric and antisymmetric part like in \begin{align*}
        T^{ij} = \frac{1}{2}\left( T^{ij} + T^{ji}  \right) + \frac{1}{2} \left( T^{ij} - T^{ji}   \right) ?
    \end{align*}

    Well, it doesn't make sense to talk about symmetries between upper and lower indicies. They are separate things. It only makes sense to compare them separately!
\end{remark}

For \(SU(N)\), we have both \(\epsilon _{i_1 \dots i_n}\) and \(\epsilon ^{i_1 \dots i_n}\). These are separate objects!

Consider for example \begin{align*}
    \epsilon _{abm}T^{abc} _{de} = \tilde{T}^c _{d e m }
\end{align*}
or \begin{align*}
    \epsilon ^{dem} T^{abc} _{de} = \tilde{T}^{abcm}
\end{align*}

We get this nice identity
\begin{align*}
    \epsilon _{ijk}\epsilon ^{ijl} = 2 \delta^{l}_k
\end{align*}
if \(i, j, k = 1, 2, 3\).

From last time, we also have the special case that \(su(2) \cong so(3)\) (since their Lie Bracket is the same). But our irreducible repr. in \(so(3)\) split into two distinct types: \begin{align*}
    &so(3): \quad 2j + 1, j = 0, \frac{1}{2}, 1, \dots\\
    &so(3): \quad 2j + 1, j = 0, 1, 2, \dots
\end{align*}
(yes, they overlap)

whereas we have \begin{align*}
    SU(2): \quad 2j + 1, j = 0, \frac{1}{2}, 1
\end{align*}

For \(SU(2)\): Only necessary to consider symmetric tensors with upper indicies!
\begin{enumerate}
    \item Bring all indicies up:\begin{align*}
        T^{i_1  \dots i_l}_{j_1 \dots j_n} \to \epsilon ^{k_1 j_1} \epsilon ^{k_2 j_2}
    \end{align*}
\end{enumerate}


\end{document}