\documentclass[a4paper]{article}
\usepackage[a4paper, margin=1in]{geometry} % Adjust margin here, e.g., 1 inch
% Some basic packages
\usepackage[utf8]{inputenc}
\usepackage[T1]{fontenc}
\usepackage{textcomp}
\usepackage[english]{babel}
\usepackage{url}
\usepackage{graphicx}
\usepackage{float}
\usepackage{booktabs}
% \usepackage{enumitem}
\usepackage{enumerate}
\usepackage[colorlinks]{hyperref}

\pdfminorversion=7

% Don't indent paragraphs, leave some space between them
\usepackage{parskip}
\usepackage{changepage}

% Hide page number when page is empty
\usepackage{emptypage}
\usepackage{subcaption}
\usepackage{multicol}
\usepackage[dvipsnames]{xcolor}

% Other font I sometimes use.
% \usepackage{cmbright}

% Math stuff
\usepackage{amsmath, amsfonts, mathtools, amsthm, amssymb}

% Add this line to make equation numbering follow section
\numberwithin{equation}{section}

% Fancy script capitals
\usepackage{mathrsfs}
\usepackage{cancel}
% Bold math
\usepackage{bm}
% Some shortcuts
\newcommand\N{\ensuremath{\mathbb{N}}}
\newcommand\R{\ensuremath{\mathbb{R}}}
\newcommand\Z{\ensuremath{\mathbb{Z}}}
\renewcommand\O{\ensuremath{\emptyset}}
\newcommand\Q{\ensuremath{\mathbb{Q}}}
\newcommand\C{\ensuremath{\mathbb{C}}}

% Easily typeset systems of equations (French package)
\usepackage{systeme}

% Put x \to \infty below \lim
\let\svlim\lim\def\lim{\svlim\limits}

%Make implies and impliedby shorter
\let\implies\Rightarrow
\let\impliedby\Leftarrow
\let\iff\Leftrightarrow
% \let\epsilon\varepsilon

% COURSE SPECIFICS
% GRIFFITHS
\ifdefined\pdfliteral
    \let\griffPdfliteral\pdfliteral
\else \def\griffPdfliteral#1{\special{pdf: literal #1}} \fi

\newcommand\griffr[1][2]{\leavevmode\hbox{\kern1pt\vbox to1ex{}\griffPdfliteral{%
    q 1 J .27 0 0 .27 0 0 cm #1 w
    0 2 m
    0 2 8.1 9.7 9.2 13.2 c
    10.4 16.8 8.4 15.4 8 14.7 c
    7.6 14 6.8 12.6 12 13 c
    17 13.5 14.5 7.8 13.7 6 c
    12.8 4.3 10.3 1.2 11.4 .2 c
    12.6 -.7 18.8 3.6 18.8 3.6 c
    18.8 3.6 l S Q
}\kern6pt}}
\newcommand\hatgriffr{\skew3\hat{\griffr[4]}}

% Add \contra symbol to denote contradiction
\usepackage{stmaryrd} % for \lightning
\newcommand\contra{\scalebox{1.5}{$\lightning$}}

% \let\phi\varphi

% Command for short corrections
% Usage: 1+1=\correct{3}{2}

\definecolor{correct}{HTML}{009900}
\newcommand\correct[2]{\ensuremath{\:}{\color{red}{#1}}\ensuremath{\to }{\color{correct}{#2}}\ensuremath{\:}}
\newcommand\green[1]{{\color{correct}{#1}}}

% horizontal rule
\newcommand\hr{
    \noindent\rule[0.5ex]{\linewidth}{0.5pt}
}

% hide parts
\newcommand\hide[1]{}

% si unitx
\usepackage{siunitx}
\sisetup{locale = FR}

% Environments
\makeatother
% For box around Definition, Theorem, ...
% \usepackage{mdframed}
\usepackage[framemethod=TikZ]{mdframed}

% Custom command to draw a rectangular border around an equation
\setlength{\fboxsep}{5pt}  % Adjust padding inside the box
\usepackage{empheq}
\newcommand*\widefbox[1]{\fbox{\hspace{1em}#1\hspace{1em}}}

\usepackage{environ}  % This package allows for easier custom environment definitions

% Define the custom environment
\NewEnviron{framed}{%
  \begin{empheq}[box=\fbox]{align}
  \BODY
  \end{empheq}
}
% Custom environment to box align equations
% \newenvironment{boxedalign}
%   {\begin{empheq}[box=\fbox]{align}}
%   {\end{align}\end{empheq}}

\newtheorem{thm}{Theorem}[subsection]
\newtheorem{defi}[thm]{Definition}
\newtheorem{lem}[thm]{Lemma}
\newtheorem{ret}{Correction}


\newtheorem*{term}{Terminology}
\newtheorem*{key}{Keywords and Related Concepts}
\newtheorem{lign}[thm]{Equation}
\newtheorem{law}[thm]{Law / Principle}

\usepackage{mathtools}
\DeclarePairedDelimiter\bra{\langle}{\rvert}
\DeclarePairedDelimiter\ket{\lvert}{\rangle}
\DeclarePairedDelimiterX\braket[2]{\langle}{\rangle}{#1\,\delimsize\vert\,\mathopen{}#2}


% \newcounter{theo}[section]
% \renewcommand{\thetheo}{\arabic{section}.\arabic{theo}}

% \mdfsetup{skipabove=1em,skipbelow=0em}
% \theoremstyle{definition}
% \newmdtheoremenv[nobreak=true]{definition}{Definition}
% \newmdtheoremenv[nobreak=true]{theorem}{Theorem}
% \newmdtheoremenv[nobreak=true]{corollary}{Corollary}
% \newmdtheoremenv[nobreak=true]{lemma}{Lemma}

% \newtheorem*{observation}{Observation}
% \newtheorem*{property}{Property}
% \newtheorem*{postulate}{Postulate}
% \newtheorem*{conclusion}{Conlusion}
% \newtheorem*{repitition}{Repitition}
% \newtheorem*{example}{Example}
% \newtheorem*{question}{Question}
% \newtheorem*{intuition}{Intuition}

% End example and intermezzo environments with a small diamond (just like proof
% environments end with a small square)
% \usepackage{etoolbox}
% \AtEndEnvironment{example}{\null\hfill$\diamond$}%
% \AtEndEnvironment{repitition}{\null\hfill$\diamond$}%
% \AtEndEnvironment{opmerking}{\null\hfill$\diamond$}%

% Fix some spacing
% http://tex.stackexchange.com/questions/22119/how-can-i-change-the-spacing-before-theorems-with-amsthm
\makeatletter
\def\thm@space@setup{%
  \thm@preskip=\parskip \thm@postskip=0pt
}


% Exercise 
% Usage:
% \oefening{5}
% \suboefening{1}
% \suboefening{2}
% \suboefening{3}
% gives
% Oefening 5
%   Oefening 5.1
%   Oefening 5.2
%   Oefening 5.3
\newcommand{\exercise}[1]{%
    \def\@exercise{#1}%
    \subsection*{Exercise #1}
}

\newcommand{\subexercise}[1]{%
    \subsubsection*{Exercise \@exercise.#1}
}

\usepackage{xcolor}
\newcommand{\textred}[1]{\textcolor{red}{#1}}

% \lecture starts a new lecture (les in dutch)
%
% Usage:
% \lecture{1}{di 12 feb 2019 16:00}{Inleiding}
%
% This adds a section heading with the number / title of the lecture and a
% margin paragraph with the date.

% I use \dateparts here to hide the year (2019). This way, I can easily parse
% the date of each lecture unambiguously while still having a human-friendly
% short format printed to the pdf.

\usepackage{xifthen}
\def\testdateparts#1{\dateparts#1\relax}
\def\dateparts#1 #2 #3 #4 #5\relax{
    \marginpar{\small\textsf{\mbox{#1 #2 #3 #5}}}
}

\def\@lecture{}%
\newcommand{\lecture}[3]{
    \ifthenelse{\isempty{#3}}{%
        \def\@lecture{Lecture #1}%
    }{%
        \def\@lecture{Lecture #1: #3}%
    }%
    \subsection*{\@lecture}
    \marginpar{\small\textsf{\mbox{#2}}}
}

\def\@chapter{}%
\newcommand{\chapter}[3]{
    \ifthenelse{\isempty{#3}}{%
        \def\@chapter{Chapter #1}%
    }{%
        \def\@chapter{Chapter #1: #3}%
    }%
    \subsection*{\@chapter}
    \marginpar{\small\textsf{\mbox{#2}}}
}

\def\@week{}%
\newcommand{\week}[3]{
    \ifthenelse{\isempty{#3}}{%
        \def\@week{Uge #1}%
    }{%
        \def\@week{Uge #1: #3}%
    }%
    \subsection*{\@week}
    \marginpar{\small\textsf{\mbox{#2}}}
}

% These are the fancy headers
% \usepackage{fancyhdr}
% \pagestyle{fancy}

% LE: left even
% RO: right odd
% CE, CO: center even, center odd
% My name for when I print my lecture notes to use for an open book exam.
% \fancyhead[LE,RO]{Gilles Castel}

% \setlength{\headheight}{5pt}

% % \fancyhead[R]{\@lecture} % Right odd,  Left even
% \fancyfoot[R]{\thepage}  % Right odd,  Left even
% \fancyfoot[C]{\leftmark}     % Center

\makeatother

% Todonotes and inline notes in fancy boxes
\usepackage{todonotes}
\usepackage{tcolorbox}

% Make boxes breakable
\tcbuselibrary{breakable}

% Usage: 
% \begin{correction}
%     Lorem ipsum dolor sit amet, consetetur sadipscing elitr, sed diam nonumy eirmod
%     tempor invidunt ut labore et dolore magna aliquyam erat, sed diam voluptua. At
%     vero eos et accusam et justo duo dolores et ea rebum. Stet clita kasd gubergren,
%     no sea takimata sanctus est Lorem ipsum dolor sit amet.
% \end{correction}
\newenvironment{correction}{\begin{tcolorbox}[
    arc=0mm,
    colback=white,
    colframe=green!60!black,
    title=Correction,
    fonttitle=\sffamily,
    breakable
]}{\end{tcolorbox}}

% Same as 'correction' but color of box is different
\newenvironment{note}{\begin{tcolorbox}[
    arc=0mm,
    colback=white,
    colframe=white!60!black,
    title=Note,
    fonttitle=\sffamily,
    breakable
]}{\end{tcolorbox}}


% Figure support as explained in my blog post.
\usepackage{import}
\usepackage{xifthen}
\usepackage{pdfpages}
\usepackage{transparent}
\newcommand{\incfig}[1]{%
    \def\svgwidth{\columnwidth}
    \import{./figures/}{#1.pdf_tex}
}

% Fix some stuff
% %http://tex.stackexchange.com/questions/76273/multiple-pdfs-with-page-group-included-in-a-single-page-warning
\pdfsuppresswarningpagegroup=1


% My name
\author{Erik Bach Ryhl}


\usepackage{slashed}

\graphicspath{ {./figs/} }

\setcounter{tocdepth}{4}
\setcounter{secnumdepth}{3}


\theoremstyle{definition} % Define theorem styles here based on the definition style (used for definitions and examples)
\newtheorem*{definition}{Definition}

\theoremstyle{plain} % Define theorem styles here based on the plain style (used for theorems, lemmas, propositions)
\newtheorem{theorem}{Theorem}[section]
\newtheorem{corollary}[theorem]{Corollary}
\newtheorem{lemma}[theorem]{Lemma}
\newtheorem{proposition}[theorem]{Proposition}
\newtheorem*{problem}{Problem}

\theoremstyle{remark} % Define theorem styles here based on the remark style (used for remarks and notes)
\newtheorem{example}[theorem]{Example}
\newtheorem*{notation}{Notation}
\newtheorem{remark}[theorem]{Remark}
\newtheorem*{solution}{Solution}



\title{Group Theory in a Nutshell for Physicists}
\begin{document}
    \maketitle
    \tableofcontents
    \newpage
    \section{Notes to self}
\subsection{To Add}
\begin{itemize}
    \item An explanation of why \begin{align*}
        \epsilon^{ijk \cdots n} R^{ip} R^{jq} R^{kr} \cdots R^{ns} = \epsilon^{p q r \cdots s} \mathrm{det}\ R  
    \end{align*}
    \item The more formal definition of a representation
\end{itemize}
\subsection{From Ludvig}
The elements of $SO(3)$ are $3D$ rotations, so the generators represented by $J_x, J_y, J_z$ generate infinitesimal rotations. Using Noether's theorem, the angular momentum operator (about some axis) in quantum mechanics is defined as the generator of infinitesimal rotations (about that axis). This is exactly the same as the Lie group! The generators correspond to operators, the exponentiated group elements correspond to actual rotations of particle. Thus, the $J_z$ angular momentum operator is (conceptually) the same as the $J_z$ in the Lie algebra of $SO(3)$. Zee writes it as $\vec{L}_{\text{operator}} = \hbar \vec{L}_{\text{Lie}}.$

Now, let $J_z \ket{m} = m \ket{m}$. Here, the $\ket{m}$ kets are the normalised eigenvectors of $J_z$, meaning they span the "eiggenspace" of the $J_z$ matrix. This definition is the same as a tensor furnishing a representation of $J_z$ in group theory; it is something, the representation of $J_z$ can be applied to by matrix multiplication from the left. Thus, tensors furnishing representations of $J_z$ are directly connected to the eigenfunctions $\ket{m}$. This means that the number of eigenvalues $m$ is equal to the dimension of the corresponding representations. Interlude:

\textit{A tensor representation of $SO(3)$ can be similarity transformad into a block-diagonal form, where the irreducible representations are furnished by traceless, symmetric tensors. They have dimensions equal to the number of independent indices of these, which for $j$ components are $2j+1$.}

Now, as the $\ket{m}$ kets transform among themselves under rotations, i.e. by $J_z$ operating on them (that's the definition of an eigenket), the corresponding representations must do the same. Therefore, the eigenkets and eigenvalues correspond to tensors furnishing \textit{irreducible} representations of the corresponding operator/representation matrix. 

For a given $j$, there are $2j+1$ corresponding $\ket{m}$ kets. Thus, if $j$ is integer, these correspond to the irreducible tensor representations of $SO(3)$ before exponentiation, or the irreducible representations of the Lie algebras. 

Now for angular momentum addition. By considering infinitesimal rotations, it is shown in Zee that $J_z \ket{j, m} \otimes \ket{j', m'} = (m+m')$. This leads to the conclusion given at the last lecture: “Taking direct products of irreducible representations of SO(3) \& SO(3) and decomposing into irreducible representations = adding angular momentum and possibly spin in QM”.

Another thing is that the Clebsch-Gordan decomposition of tensor representations yields the same possible values as the the possible values of $m$ are in quantum mechanics:
\begin{align}
	j \otimes j' = (j+j') \oplus (j+j'-1) \oplus \dots \oplus (j-j'), \quad \text{for } j > j'.
\end{align}

    % \section{Lecture 1}
    % \section{Postulater}
\begin{enumerate}[i)]
    \item En partikel er beskrevet ved en \textit{bølgefunktion} \(\Psi (\mathbf{x}, t)\).
    \item \(\Psi (\mathbf{x}, t)\) udivkler sig ved Schrödinger-ligningen \begin{align*}
        \boxed{i \hbar \frac{\partial \Psi }{\partial t} = \left(  - \frac{\hbar ^{2} }{2m} \nabla^{2}  + V(\mathbf{x}, t) \right) \Psi (\mathbf{x},t).}   
    \end{align*}
    \item Hvis partiklen er beskrevet ved bølgefunktionen \(\Psi (\mathbf{x}, t)\) så har vi \begin{align*}
        \boxed{P_{\mathcal{V} } = \int_{\mathcal{V}}  d^3 x^{\prime}\  \Psi^{\ast} (\mathbf{x},t) \Psi (\mathbf{x} ,t).}
    \end{align*}
    \item Hvis vi til tiden \(t^{\prime} \) måler positionen og detekterer partiklen ved \(\mathbf{x} = \mathbf{x}^{\prime}\) så \textit{kollapser} bølgefunktionen \begin{align*}
        \boxed{\Psi (\mathbf{x}, t) \longrightarrow \delta (\mathbf{x} - \mathbf{x}^{\prime} ).}
    \end{align*}   
    \item Impulsoperatoren er givet ved \begin{align*}
        \boxed{\hat{p} = -i \hbar \frac{\partial}{\partial x}.}
    \end{align*}
\end{enumerate}

\section{Den Tidsuafhængige Schrödinger-Ligning}
Vi starter fra Schrödinger-ligningen (SE): \begin{align*}
    i \hbar \frac{\partial \Psi (x, t)}{\partial t} = \left( - \frac{\hbar ^{2} }{2m} + V(x, t)  \right) \Psi (x, t)
\end{align*}
Hvad repræsenterer leddet \begin{align*}
    \frac{1}{2m}\left( -\hbar ^{2} \frac{\partial^{2} }{\partial x^{2} }  \right) 
\end{align*}
egentlig? Vi genkender leddet som \begin{align*}
    \frac{1}{2m} \hat{p}^{2} 
\end{align*}
og ser at det ligner en kvantemekanisk analog til kinetisk energi. Vi ser altså, at jo mere energi vi har (højresiden), jo hurtigere må tidsændringen af vores bølgefunktion være. Det giver faktisk meget god mening!

\textbf{Forventningsværdien for en operator} 
\begin{align*}
    \left\langle A \right\rangle = \int _{- \infty} ^{\infty} d x \  \Psi (x, t)^{\ast} A \Psi (x, t)
\end{align*} 

\textbf{Spredningen for en operator} 
\begin{align*}
    \sigma _A ^{2} = \left\langle A^{2}  \right\rangle - \left\langle A \right\rangle ^{2} 
\end{align*}
\textbf{Dispersion (afstand fra middelværdien)}
\begin{align*}
    \Delta A = A - \left\langle A \right\rangle 
\end{align*} 

\subsection{Løsning af den Tidsuafhængige SE}
Det antages at \(V(x, t) = V(x)\) (det er dét, der menes med tidsuafhængig). Vi håber at vi kan separere ligningen (da den er 2. ordens) og prøver ansatzen \begin{align*}
    \Psi (x, t) \overset{?}{=} \psi (x) \phi (t).
\end{align*}

Det ses direkte fra denne ansatz at \begin{align*}
    i \hbar  \frac{1}{\phi (t)} \frac{\partial \phi (t)}{\partial t} = - \frac{\hbar ^{2} }{2m} \frac{1}{\psi (x)} \frac{\partial ^{2} \psi (x)}{\partial x ^{2} } + V(x) .
\end{align*}

Dermed må begge sider være lig (\textred{samme?}) en konstant. Det ses også at begge sider har enheden Joule, og dermed vælger vi at kalde konstanten på højresiden \(E\): \begin{align*}
    i \hbar \frac{1}{\phi (t)} \frac{\partial \phi (t)}{\partial t} = E
\end{align*}  
hvilket giver os en roterende fase-faktor som løsning:
\begin{align*}
    \boxed{\phi (t) = e^{- i Et / \hbar }}
\end{align*}
og som forventet: jo højere energi, jo hurtigere ændrer den tidsafhængige del af \(\Psi (x, t)\) sig!

Faktisk er studiet af højresiden \begin{align*}
    - \frac{\hbar ^{2} }{2m} \frac{1}{\psi (x)} \frac{\partial ^{2} \psi (x)}{\partial x ^{2} } + V(x) = E
\end{align*}
så vigtig i sig selv, at ganger vi igennem med \(\psi (x)\) får vi det, vi kalder den tidsuafhængige SE \begin{align*}
    \boxed{\left[ -\frac{\hbar ^{2} }{2m} \frac{\partial ^{2} }{\partial x ^{2} } + V(x) \right] \psi (x) = E \psi (x)}
\end{align*} 
Resten af KM1 vil hovedsageligt beskæftige sig med at løse denne ligning for forskellige potentialer. Betragter vi dét, der agerer på \(\psi (x)\) på højresiden af ligningen som en operator, så genkender vi det som en egen"vektor" ligning. Vi kalder derfor \(\psi (x)\) en \textit{egenfunktion} for operatoren. Disse egenfunktioner er altså løsningerne for den tidsuafhængige SE, og energiniveauerne er egenværdierne. 

Men idet vi da har et uendeligt-dimensionelt vektorrum (vores operator har uendeligt mange indgange), så forventer vi også uendeligt mange løsninger. Vi skriver derfor også (for at indeksere dem) \begin{align*}
    \left[ - \frac{\hbar ^{2} }{2m} \frac{\partial ^{2} }{\partial x ^{2} } + V(x)\right] \psi_n (x) = E_n \psi_n (x)
\end{align*}

\subsection{Infinite Square Well}
Hvis vores potentiale er givet ved
\begin{align*}
    V(x) = 
    \begin{cases}
    0, & 0 \le x \le a, \\
    \infty, & \text{otherwise}.
    \end{cases}
\end{align*}

får vi \begin{align*}
    - \frac{\hbar ^{2} }{2m} \frac{\partial ^{2} \psi_n (x)}{\partial x ^{2} } = E_n \psi _n (x)
\end{align*}
sådan at \begin{align*}
    \psi _n (x) = A \sin (kx) + B \cos (kx)
\end{align*}

Med vores boundary conditions ser vi først at \begin{align*}
    \psi_n(x) = A \sin (kx)
\end{align*}
og dernæst at \begin{align*}
    k_n = \frac{n \pi }{a} 
\end{align*}

% \section*{Mathematical Interlude: Gaussian Integrals}



    \newpage

    \section{Tensors and Representations of the Rotation Group \(SO(N)\)}
\begin{definition}[Representation (take 1)]
    A representation is homomorphic map \(D^{(n)} : \mathcal{G} \to GL (n, \mathbb{C})\). That is, a map from the group \(\mathcal{G}\) to the set of all linear transformations on a vector space \(V\) with \(\mathrm{dim} (V) = n\). That the map is homomorphic means that it \textit{preserves structure}: \begin{align*}
        D(g_1) D(g_2) = D(g_1 g_2)
    \end{align*}
    This means that any product between the representations of group elements (in the vector space) is equal to the representation of the products of the group elements (in the group).
\end{definition}

\begin{definition}[Direct Product]
    Given two mathematical objects \(A\) and \(B\) carrying any number of free indicies, we define the direct product between them by defining how to obtain its components:
    \begin{align*}
        (A \otimes B)_{} = A_{i, j, k, \cdots n} \otimes B_{p, q, r, \cdots s} \coloneqq 
    \end{align*}
\end{definition}

\subsection{The Special Unitary Groups \(SU(N)\)}
\begin{definition}[\(SU(N)\)]
    \(SU(N)\) are all the \(N \times N\) matrices with complex entries satisfying that \begin{align*}
        U ^{\dagger} U = 1, \quad \mathrm{det}\ U = 1
    \end{align*}
\end{definition}
The defining representation is defined by being furnished by complex vectors with \(N\) entires. Unitary matrices preserve inner products of the type \(v ^{\dagger} w\), since 
\begin{align*}
    v^{\prime \dagger} w^{\prime} = \left( U v \right) ^{\dagger} \left( U w \right) = v ^{\dagger} U ^{\dagger} U w = v^{\dagger} w
\end{align*} 
Also, we see that \begin{align*}
    1 = \mathrm{det} U ^{\dagger} U = \mathrm{det} U ^{\dagger} \mathrm{det} U = \mathrm{det} U ^{\ast} \mathrm{det} U = \left| \mathrm{det} U \right| ^{2} 
\end{align*}
which implies that \(\mathrm{det} U = e^{i \theta }\). We choose by convention that \(\mathrm{det}  U = 1\) as defined above. 

It turns out that \(SU(2)\) describes both fermions (spin) as well as weak interactinos (isospin). It is also the case that \(SU(3)\) describes QCD (quark interactions). The really cool idea is that (with coupling constants), the standard model is described by \begin{align*}
    SU(3) \otimes SU(2) \otimes U(1)
\end{align*}

Since \begin{align*}
    SU(3) \otimes SU(2) \otimes U(1) \subset SU(5)
\end{align*}
an attempt was made at a "Great Unified Theory" based on the symmetries of this group, but it hasn't worked out (and probably won't). 

One can also do expansions in \(SU(N)\) like:
\begin{align*}
    \left\langle O \right\rangle = O_0 + \frac{1}{N^{2} } O_1 + \frac{1}{N_4} O_2 + \cdots 
\end{align*}

\subsubsection{\(SU(N)\) as a Lie Group with Lie Algebra (\(su(N)\))}
To be a Lie Group, we should be able to expand our group element as an exponential map around the identity\begin{align*}
    g = \exp \left( i \sum_{a} \theta  ^a  T^ a\right) 
\end{align*}
where \(\theta ^a\) are the infinitesimal parameters, and \(T^a\) are the generators.

The Lie Bracket of the Algebra is a bilinear, antisymmetric product which needs to satisfy that \begin{align*}
    \left[ T^a, T^b \right] = i f^{abc} T^c
\end{align*}
as well as the Jacobi Identity.

Taking an element in our group, \(U = g = e^{i \epsilon h}\), we find that \begin{align*}
    1 = U^{\dagger} U = \left( 1 - i \epsilon h ^{\dagger} + O(\epsilon ^{2} ) \right)\left( 1 + i \epsilon h ^{\dagger} + O(\epsilon ^{2} ) \right) = 1 + i \epsilon (h - h ^{\dagger} + O(\epsilon ^{2} ))
\end{align*} 
By equating terms of like order, we see that our generators need to be Hermitian\begin{align*}
    h = h^{\dagger} 
\end{align*}

We see that \begin{align*}
    \det(U) = 
    \begin{vmatrix}
    1 + i \epsilon h_{11} &  i \epsilon h_{12}  & \cdots & i \epsilon h_{1n} \\
    i \epsilon h_{21} & 1 + i \epsilon h_{22}  & \cdots &  \\
    \vdots & \vdots & \ddots & \vdots \\
     &   & \cdots & 1 + i \epsilon h_{nn}
    \end{vmatrix}
\end{align*}
Keeping only the terms of first order in \(\epsilon\), we obtain \begin{align*}
    \det(U) = 1 + i \epsilon Tr(h) + O(\epsilon ^{2} )
\end{align*} 
which forces \(Tr(h) = 0\) 
\textred{Review this!}

Counting the number of generators in \(SU(N)\), we have \(N^{2} \) complex and \(N^{2} \) real components. But \(h ^{\dagger}  = h\) enforces a constraint, and the \(Tr(h) = 0\) also removes a single free component. Thus we have \begin{align*}
    N^{2} -1
\end{align*}
generators in \(SU(N)\).

\begin{example}
    \(su(3)\) has 8 generators written in terms of the Gell-mann Matrices \(\lambda ^1, \lambda ^2, \dots, \lambda ^8\). \begin{align*}
        T^a = \frac{\lambda ^a}{2}
    \end{align*}
    while we have \begin{align*}
        \left[ T^a, T^b \right] = i f^{abc} T^c
    \end{align*}
\end{example}

\subsection{Tensors for \(SU(N)\)}
In the defining repr., we have \begin{align*}
    v ^i \to (v^{\prime} )^i = U^i _j v^j
\end{align*}
Now, we find that the tensor representation will look like \begin{align*}
    T^{ij} \to \left( T ^{\prime}  \right) ^{i j} = U ^i _k U^j _l T^{kl}
\end{align*}

Or, higher rank: \begin{align*}
    T^{i_1 \dots i_n} \to \left( T^{\prime}  \right) ^{i_1 \dots i_n} = U^{i_1} _ {j_1} \cdots U^{i_n} _ {j_n} T^{j_1 \dots j_n}
\end{align*}
If we can find non-trivial subspaces in the product representation (which is furnished by these tensors), then the tensor repr. is reducible. This is equivalent to our tensors having specific symmetry properties. We find that \begin{align*}
    T^{ii} \to \left( T^{\prime}  \right) ^{ii} = U^i _k U^i _l T^{k l} = \left( U^T \right) ^k _i \left( U \right) ^i _ l T^{kl} 
\end{align*}
Thus unlike \(SO(N)\), the trace doesn't furnish an irreducible representation in \(SU(N)\), since \(U^T U \neq 1\) . Any two-rank tensor can thus be decomposed into two parts (constrast to three):
\begin{align*}
    T^{ij} = \frac{1}{2}\left( T^{ij} + T^{ji}  \right) + \frac{1}{2} \left( T^{ij} - T^{ji}   \right) 
\end{align*}

\subsection{The Conjugate Representation}
From a repr. \(D(g)\) we know that \(D^{\ast} (g)\) is a repr. as well (the conjugated repr.).

Now, consider \begin{align*}
 \left( v^i \right) ^{\ast} \to \left( U ^i _j \right)^{\ast} \left( v^j \right)^{\ast} = \left( U^{\dagger}  \right) ^j _i  \left( v^j \right) ^{\ast} 
\end{align*}
We now define an object that carries the conjugated repr. as an object that transforms like this \begin{align*}
    w_i \to \left( U^{\dagger}  \right) ^j _i w_j = w_j \left( U ^{\dagger}  \right) ^j _i
\end{align*}
where we use the lower index to denote these types of objects. The last equality is a reordering (these are components) to remind us that we want the matching indicies (upper and lower). We see that the \textit{components} of the vectors in the adjoint representation transforms like the basis vectors themselves in the defining representation. 

In \(SO(N)\), our components transform like \begin{align*}
    v^i \to R^i _j v^j
\end{align*}
whereas the basis vectors transform like \begin{align*}
    e^i \to R^j _ i v^j
\end{align*}

In the adjoint repr., the components transform like the basis vectors in \(SO(N)\)!

We then get \begin{align*}
    T_{i_1 \dots i_ n} \to T^{\prime} _{i_1 \dots i_n} = T_{j_1 \dots j_n} U^{\dagger j_1}_{i_1} \cdots U^{\dagger j_n}_{i_n} 
\end{align*}

And we can also combine!\begin{align*}
    T^{j_1 j_2}_{i_1 i_2} \to U^{j_1}_{l_1} U^{j_2} _{l_2} T^{l_1 l_2} _{k_1 k_2} U^{\dagger k_1}_{i_{1} } U^{\dagger k_2}_{i_2}  
\end{align*}

We can now take a generalized trace, which \textit{will} be invariant under an \(SU(N)\) transformation: \begin{align*}
    T^i _ i \to \left( T^{\prime}  \right) ^i _i = U^i _k T ^k _ l U^{\dagger l }_ i = \left( U^{\dagger} \right)^l _ i U ^i _k T^k _l = \delta_{kl} T ^k _ l = T^k _ k  
\end{align*}
An invariant traceeeeeee!

We can thus decompose \begin{align*}
    T^i _j = \underbrace{T^i _j - \frac{1}{N}\delta^i _ j T^k _k}_{irreducible} + \frac{1}{N}\delta ^i _ j T ^k _ k 
\end{align*}

Irreducible tensors: Specific symmetries in upper and lower indicies seperately. And they are traceless.

\begin{remark}
    One might think: "Where's the symmetric and antisymmetric part like in \begin{align*}
        T^{ij} = \frac{1}{2}\left( T^{ij} + T^{ji}  \right) + \frac{1}{2} \left( T^{ij} - T^{ji}   \right) ?
    \end{align*}

    Well, it doesn't make sense to talk about symmetries between upper and lower indicies. They are separate things. It only makes sense to compare them separately!
\end{remark}

For \(SU(N)\), we have both \(\epsilon _{i_1 \dots i_n}\) and \(\epsilon ^{i_1 \dots i_n}\). These are separate objects!

Consider for example \begin{align*}
    \epsilon _{abm}T^{abc} _{de} = \tilde{T}^c _{d e m }
\end{align*}
or \begin{align*}
    \epsilon ^{dem} T^{abc} _{de} = \tilde{T}^{abcm}
\end{align*}

We get this nice identity
\begin{align*}
    \epsilon _{ijk}\epsilon ^{ijl} = 2 \delta^{l}_k
\end{align*}
if \(i, j, k = 1, 2, 3\).

From last time, we also have the special case that \(su(2) \cong so(3)\) (since their Lie Bracket is the same). But our irreducible repr. in \(so(3)\) split into two distinct types: \begin{align*}
    &so(3): \quad 2j + 1, j = 0, \frac{1}{2}, 1, \dots\\
    &so(3): \quad 2j + 1, j = 0, 1, 2, \dots
\end{align*}
(yes, they overlap)

whereas we have \begin{align*}
    SU(2): \quad 2j + 1, j = 0, \frac{1}{2}, 1
\end{align*}

For \(SU(2)\): Only necessary to consider symmetric tensors with upper indicies!
\begin{enumerate}
    \item Bring all indicies up:\begin{align*}
        T^{i_1  \dots i_l}_{j_1 \dots j_n} \to \epsilon ^{k_1 j_1} \epsilon ^{k_2 j_2}
    \end{align*}
\end{enumerate}


\end{document}